\textbf{O surgimento da noite} relata, através de narrativas, o surgimento de elementos do mundo dos Yanomami. Da noite, como diz o título, mas também do tabaco, do cipó, da banana, entre outros. Tudo acontece através do personagem Horonamɨ, um grande pajé que surgiu dele mesmo e junto com as florestas, e ensinou aos Yanomami como morar nelas. Além de compartilhar os conhecimentos com o próprio povo, também o fez com os estrangeiros. \textit{O surgimento da noite} faz parte do segmento Yanomami da coleção Mundo Indígena --- com \textit{O surgimento dos pássaros}, \textit{A árvore dos cantos} e \textit{Os comedores de terra} ---, que reúne quatro cadernos de histórias dos povos Yanomami, contadas pelo grupo Parahiteri. Trata-se da origem do mundo de acordo com os saberes deste povo, explicando como, aos poucos, ele veio a ser como é hoje.

\textbf{Anne Ballester} foi coordenadora da \textsc{ong} Rios Profundos e conviveu vinte anos junto aos Yanomami do rio Marauiá. Trabalhou como professora na área amazônica, e atuou como mediadora e intérprete em diversos \textit{xapono} do rio Marauiá --- onde também coordenou um programa educativo. Dedicou-se à difusão da escola diferenciada nos \textit{xapono} da região, como à formação de professores Yanomami, em parceria com a \textsc{ccpy}\,Roraima, incorporada atualmente ao Instituto Socioambiental (\textsc{isa}). Ajudou a organizar cartilhas monolíngues e bilíngues para as escolas Yanomami, a fim de que os professores pudessem trabalhar em sua língua materna. Trabalhou na formação política e criação da Associação Kurikama Yanomami do Marauiá, e participou da elaboração do Plano de Gestão Territorial e Ambiental (\textsc{pgta}), organizado pela Hutukara Associação Yanomami e o \textsc{isa}.

\textbf{Mundo Indígena} reúne materiais produzidos com pensadores de diferentes povos indígenas e pessoas que pesquisam, trabalham ou lutam pela garantia de seus direitos. Os livros foram feitos para serem utilizados pelas comunidades envolvidas na sua produção, e por isso uma parte significativa das obras é bilíngue. Esperamos divulgar a imensa diversidade linguística dos povos indígenas no Brasil, que compreende mais de 150 línguas pertencentes a mais de trinta famílias linguísticas.



