\textbf{O surgimento da noite} reúne narrativas que abordam o surgimento de outros elementos
do mundo natural e social dos Yanomami. São narrados o surgimento do tabaco, do cipó e da banana através
das aventuras do personagem Horonamɨ. Horonamɨ é um grande pajé que surgiu de si mesmo, 
assim como é relatado na narrativa. Surgiu junto com as florestas e ensinou aos Yanomami 
como morar nelas. Além de compartilhar os conhecimentos com o povo, ele também
compartilhou suas histórias com os estrangeiros.

\textbf{Anne Ballester} nasceu em 1955 na França viveu por vinte e quatro anos com os Yanomami. 
Enquanto ativista, trabalhou como agente de saúde no combate à malária, foi alfabetizadora em língua 
yanomami e professora de português para jovens e adultos em posições de liderança indígena. É cofundadora da \textsc{ong} Rios Profundos. Atuou como tradutora e organizadora dos livros \textit{A árvore dos cantos}, \textit{O surgimento dos pássaros}, \textit{O surgimento da noite} e \textit{Os comedores de terra}, todos incluídos na Coleção Mundo Indígenas.

\textbf{Coleção Mundo Indígena} reúne materiais produzidos com pensadores de diferentes povos indígenas e pessoas que pesquisam, trabalham ou lutam pela garantia de seus direitos. Os livros foram feitos para serem utilizados pelas comunidades envolvidas na sua produção, e por isso uma parte significativa das obras é bilíngue. Esperamos divulgar a imensa diversidade linguística dos povos indígenas no Brasil, que compreende mais de 150 línguas pertencentes a mais de trinta famílias linguísticas.



