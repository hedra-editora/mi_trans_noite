\chapterspecial{O {surgimento} {da} {noite}}{}{}
 

 

\letra{H}{oronamɨ procurou} aquilo que nos permite dormir. Ele fez aquilo que nos
fará dormir. Aconteceu em toda a floresta. Ele procurou sem desistir,
procurou, procurou e acabou encontrando essa coisa perto da sua moradia.
A cauda da coisa já estava visível, pendurada em um galho, mas Horonamɨ
pensava que a coisa estaria sentada na raiz de uma árvore e continuou
procurando longe, em todas as direções. 

Não foi a noite que surgiu sozinha, de repente, para nós dormirmos.
Assim, quem fez não foi outro. Não foi outro que fez anoitecer: foi
Horonamɨ, e apenas Horonamɨ, quem soprou nosso sono --- somente ele. 

Qual a razão dessa procura? Como de dia ninguém parava de fazer sexo ---
vocês também não fazem sexo de dia? --- e como a noite não existia ---
era sempre luz forte do dia --- para ele esquecer os outros fazendo
sexo, ele procurou a noite para envolver todos na escuridão. 

A noite estava empoleirada em cima de uma árvore não muito distante.
Parecia com um mutum empoleirado, cuja cauda repousava na parte alta de
um galho inclinado de uma árvore \emph{paikawa}.\footnote{  Árvore baixa, chamada localmente de pé"-de"-maçarico.}  Assim
era a escuridão. Apesar de a noite parecer um mutum, Horonamɨ conseguiu
encontrá"-la. A noite também cantava como um mutum. 

Nessa época, os animais --- como arara, mutum, queixada, anta, veado,
caiarara, maitaca, irara, tamanduá"-bandeira, papagaio e jabuti --- eram
Yanomami e, como os Yanomami, moravam em xapono. Horonamɨ designou cada
espécie de animal e deu"-lhes seus nomes. Naquela época, ele procurou
pela terra firme sem descanso, quando não havia xaponos espalhados pela
selva; havia somente o xapono dele.\footnote{  Horonamɨ realiza diversas buscas para encontrar tudo que os Yanomami
usam para viver.}  Os animais também
viviam em xapono.\footnote{  Isto é, eram gente.} 

Quando Horonamɨ soprou a escuridão com sua zarabatana para nós
dormirmos, ele queria que anoitecesse. Ele encontrou a escuridão e
soprou. Depois de fazer cair a escuridão, ao mesmo tempo se desenhou um
pequeno círculo no chão, embaixo do lugar onde estava empoleirado o dono
da escuridão.

O pai do cunhado de Horonamɨ se chamava Manawë. Ele era uma boa pessoa,
e avisou: 

 --- Ele vai achar agora! Tomem cuidado! --- avisou Manawë no xapono. 

Quando Horonamɨ flechou o mutum da noite, apesar de estar perto da sua
moradia e de retornar correndo, ele também sofreu, porque anoiteceu
de uma vez. Depois de ter soprado a noite em todos os cantos, e de ter
corrido, ele adormeceu. Naquela noite, os Yanomami também sofreram. Não
anoiteceu devagar. Até Horonamɨ passou fome, pois não tinha como fazer
fogo. Ele acabou ficando na escuridão, apesar de estar perto do seu
xapono. Como foi assim que aconteceu, a mãe dele também sofreu, todos
ficaram tontos de fome à noite. A escuridão perseguiu Horonamɨ bem de
perto, e ele estava com fome. 

Depois de a noite apagar o dia, os que moravam com ele morreram de
fome, pois comiam somente terra, comiam terra vorazmente e sofriam. Não
sobreviveram. Até seu próprio cunhado sofreu e quase morreu. Horonamɨ
ficou angustiado.

Havia então três pajés: o avô, o avô mais novo e o cunhado, e eles
esquartejaram a noite, fazendo reaparecer a luz do dia. 

Para as pessoas não comerem mais terra, Horonamɨ foi caçar. Ele nos
ensinou a caçar. Ele tinha uma zarabatana, que alguns Yanomami usam para
soprar, era isso que ele usava. Ele soprava os animais, tinha um sopro
forte, e foi assim que ele nos ensinou a matar a caça com veneno. 

 É assim, é a própria história dos antepassados. É a história
daquele que se apossou da floresta, é o início de tudo, a história do
primeiro dono da floresta, Horonamɨ.
