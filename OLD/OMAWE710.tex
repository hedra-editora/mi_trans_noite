\chapterspecial{{Omawë}}{}{}
 

 

\letra{E}{sta história} começa com o nome dos Hoaxiwëteri. Omawë e Yoasiwë moravam
com os Hoaxiwëteri. O tuxaua Hoaxi convivia com eles, por isso se
chamavam Hoaxiwëteri. Nesse mesmo lugar, junto com os Hoaxiwëteri,
moravam Omawë, que era o irmão mais novo, e Yoasiwë, o mais velho.
Omawë, mais novo, nasceu depois de Yoasiwë. Então, Yoasiwë e seu irmão
mais novo, Omawë, moravam com os Hoaxiwëteri.

Eles pegaram a filha do monstro Raharariwë. Os dois viram a filha de
Raharariwë. Totewë, outro nome de Yoasiwë, viu a filha de Raharariwë
sentada, pegando piabas --- ensinando assim a pegar piabas. Yoasiwë
desceu ao rio e, apesar de ele não ter anzóis, onde estava pegando
piabas, ele fez aparecer um tipo de anzol. Não existia anzol, mas com
sua mente, ele o fez surgir e o amarrou em uma espécie de gancho de pau.

Raharariwë e suas filhas moravam no rio Tanape. Era lá que ficava o
xapono de Raharariwë. Omawë as encontrou no rio Tanape. 

Era a própria filha de Raharariwë que se chamava Tepahariyoma, que é o
nome de um tipo de matrinxã, aquele peixe branco, de rosto bonito. A irmã mais nova se chamava Peixe. 

No início, quando não havia mulher nos outros xapono, quando não existia
mulher entre os homens, os dois pegaram e levaram a filha mais velha de
Raharariwë. O irmão mais novo, que era lindo, conseguiu pegá"-la, embora
os dois fossem anambés"-azuis muito bonitos. 

--- Que passarinho bonito! --- Vocês dizem assim, pois Omawë era bonito.
Foi para ele que a mulher se entregou. 

O irmão mais velho se chamava Hëɨmɨriwë, queria se transformar em
anambé"-azul\footnote{  N\,O anambé"-azul fornece penas azuis usadas para fazer os brincos dos
pajés.} ; o nome do irmão mais novo era Omawë,
saíra"-paraíso, bonito, para conseguir pegar as duas mulheres, foi ele
quem fez as duas mulheres se levantarem. Depois de os dois pegarem essas
duas mulheres, não perguntem o que aconteceu! 

Omawë e seu irmão mais velho, Yoasiwë, o menos bonito, não foram ao rio,
pois estavam em um lugar diferente. Não moraram logo no lugar onde
pegaram as filhas do monstro aquático Raharariwë: foi a imagem deles que
se deslocou, na forma de passarinho.

Omawë não andava na terra, a história de Omawë é essa, que ele não
andava na terra para pegar mulher, pois queria se tornar espírito. Ele
não criou os Yanomami. Ele era sozinho, independente, pois queria se
tornar eterno. A imagem dele ainda chega aos Yanomami, é ele que se
chama Omawë.
