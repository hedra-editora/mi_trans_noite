\chapterspecial{A {vingança} {de} {horonam}ɨ}{}{}
 

\letra{O}{nde} o sol se põe, naquela parte da floresta, foi naquela parte que se
transformaram. Os Cuatás viviam como Yanomami. Eles são Yanomami.
Moravam como nós, na terra plana. Pica"-Pau Vermelho morava junto com os
Cuatás, os \emph{Rapoahiteri} e Lagartixa. Pica"-Pau Vermelho e
Lagartixa, os salvadores de Horonamɨ, moravam com os \emph{Rapoahiteri}.
Horonamɨ os encontrou, ele mesmo. Ele os viu comendo. Eles comiam abios.
Eles ardilosamente o chamaram para fazê"-lo subir; provocaram o encontro
para fazê"-lo gritar, chamando"-o.

Horonamɨ subiu, eles o fizeram subir, subir… Eles o chamaram e
quando ele subiu e chegou bem no centro da árvore, em vez de comer, eles
puxaram outra árvore, como se estivesse amarrada, e enquanto desciam por
ela, disseram a Horonamɨ:

--- Fique aí comendo! Aí você ficará satisfeito! Coma essas frutas que
ninguém pegou! --- disseram eles.

Eles fizeram com que Horonamɨ ficasse ali. Cuatá, do abieiro onde
estava, puxou a árvore taxizeiro com o fio esticado e a ponta mal
encaixada, segurando"-a somente pelas folhas. Todos os Cuatás saíram, e
aquele que eles tinham chamado ficou sozinho. Eles o deixaram preso no
abieiro. O taxizeiro deu um impulso. Ficou só o abieiro. 

Horonamɨ ficou agoniado, mesmo sendo Horonamɨ. Ficou gritando de cima,
ficou gritando, ficou preso lá. Quem iria buscá"-lo? 

--- Quem virá me buscar? --- pensava ele, chorando. 

Agoniado, estava muito triste, gritava e pedia socorro;
os Rapoahiteri\emph{ }e os Cuatás o deixaram naquela
situação\emph{.} Esse é o nome dos primeiros habitantes, os que viviam
na mesma época que os primeiros Yanomami, Rapoahiteri. Foram eles que o
deixaram ali agoniado. 

Bem depois, Lagartixa escutou os gritos de Horonamɨ. Lagartixa subiu,
para fazê"-lo descer, queria buscá"-lo, queria carregá"-lo nas suas costas,
mas ele recusou, receando escorregar com ele. Horonamɨ estava com medo
de descer de cabeça para baixo com Lagartixa. 

--- Não, você não vai me fazer descer direito --- disse
Horonamɨ --- Você vai me fazer cair! 

--- Vamos tentar! --- disse Lagartixa --- Não tenha medo, eu não vou te
fazer cair! Eu te seguro bem forte! Coloque suas mãos, assim! 

Apesar de Lagartixa dizer isso, Horonamɨ tentou, mas os dois ficaram de
cabeça para baixo. Ele gritava, quase caiu de cima, Lagartixa quase o
fez cair e, como não dava certo, Horonamɨ desistiu. Quando ele desistiu,
porque infelizmente não dava certo, Pica"-Pau Vermelho escutou a voz de
Horonamɨ: 

--- De quem é essa voz, de quem é essa voz? Parece a voz de alguém em
dificuldade --- disse --- Alguém parece estar sofrendo mesmo, a voz,
qual é seu problema, ɨ̃ɨɨ? --- disse. 

O Pica"-Pau Vermelho chegou até lá e fez uma série de buracos, fez uma
espécie de escada no tronco da árvore. Ele fez a linha de buracos chegar
certinho à forquilha da árvore onde estava Horonamɨ. Ele mandou: 

--- Vai! Coloque suas mãos nos buracos e desça. Você não vai cair! Os
buracos estão prontos! --- disse. 

Ele fez muitos buracos, Pica"-Pau Vermelho. Foi ele quem resolveu o
problema. Esses pica"-paus são os que fazem buracos nas árvores. Foi ele
quem fez Horonamɨ descer, tirando"-o daquela situação. É o nome dele
mesmo, Pica"-Pau Vermelho\footnote{  Toromɨm no original.}, ele era Yanomami. Graças à
ação dele, os nossos antepassados se reproduziram e se multiplicaram.
Foi assim. Pica"-Pau Vermelho não era um animal, era um Yanomami. Ele
existia como Yanomami e foi ele que fez Horonamɨ descer. 

--- Não responda mais, nem sempre você encontrará sempre alguém para te
ajudar, tome outro rumo quando alguém te chamar! --- ele aconselhou a
Horonamɨ. 

--- Sim, você é meu amigo, eu gosto mesmo de você, vou te proteger, eu
não vou te fazer mal! --- agradeceu Horonamɨ. 

Depois disso, como vingança, Horonamɨ estragou nossos alimentos, ele nos
fez comer alimentos amargos, ele tornou os alimentos estranhos, nos
anestesiou a boca para os alimentos comestíveis, fez nosso paladar
estranhar outros alimentos. Ele enfiou uma flechinha envenenada nos
alimentos, enfiou em todos. Kuku o comeu porque ele agiu assim, ele
estragou todos os alimentos dessa forma. Ele os tornou amargos. 

No início, eles comiam cabaris crus, quando eram saborosos, pois não
eram amargos, antes de ele os envenenar. Eles os comiam e eram gostosos,
assim, comiam cabaris gostosos como beiju. Simplesmente os cozinhavam;
cozidos, os cabaris eram comidos no mesmo instante. Apesar de eles serem
assim, depois de Horonamɨ os picar com a flechinha --- ele picou todas
as sementes das frutas com o veneno --- ele os tornou amargos, todos.
Foi o que ele fez, e mostrou para eles. Ninguém mostrou a ele os frutos
amargos, foi ele que os tornou amargos, picando"-os com veneno. Quando
ele terminou de picar todas as frutas, ele avisou: 

--- Isto que vocês comem são cabaris --- disse Horonamɨ. --- São
cabaris, e eram bons, mas vocês não os prepararão mais como preparavam.
Depois de algumas noites, vocês pisarão em cima deles e eles ficarão sem
gosto, aí vocês vão buscá"-los, vocês os comerão. Agora eles são
amargos! --- disse ele --- A preparação vai demorar muitos
dias! --- acrescentou. 

Assim fizeram. Aconteceu. Não foi qualquer pessoa que estragou as
frutas, foi Horonamɨ quem surgiu primeiro e as estragou, não foi um
descendente dele. Depois de fazer isso, estragar os alimentos com o
veneno, a seta com veneno estragou a cutia, grudou ao rabo da cutia, e
está lá ainda. O rabo das cutias se tornou a seta da zarabatana de
Horonamɨ, foi o que ele fez, e a cutia sofreu muito. Horonamɨ fez isso
com todos os alimentos, aqueles que eram gostosos, as frutas que eram
gostosas. 

--- Não sobrou nenhum!

De fato, nenhum sobrou mesmo, por isso ele disse assim. Ele os picou com
um veneno muito amargo. Quando ele terminou, ele foi se acabar também:
Kuku o comeu. Assim que foi. Como Horonamɨ não ficava quieto, ele acabou
numa situação difícil. 
