\chapter{Para ler as palavras yanomami}


Foi adotada neste livro a ortografia elaborada pelo linguista Henri Ramirez, que é a mais utilizada no Brasil e, em particular, nos programas de alfabetização de comunidades yanomami. 

\begin{itemize}
\item[/ɨ/] vogal alta, emitida do céu da boca, e que soa próximo a I e U
\item[/ë/] vogal entre o E e o O do português
\item[/w/] U curto, como em “língua”
\item[/y/] I curto, como em “Mário”
\item[/e/] vogal E, como em português
\item[/o/] O, como em português
\item[/u/] U, como em português
\item[/i/] I, como em português
\item[/a/] A, como em português
\item[/p/] como P ou B em português
\item[/t/] como T ou D em português
\item[/k/] como C de “casa”
\item[/h/] como o RR em “carro”, aspirado e suave
\item[/x/] como X em “xaxim”
\item[/s/] como S em “sapo”
\item[/m/] como M em “mamãe”
\item[/n/] como N em “nada”
\item[/r/] como R em “puro”
\end{itemize}