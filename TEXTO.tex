\chapter{O surgimento da noite}

\letra{H}{oronamɨ} procurou aquilo que nos permite dormir. Ele fez aquilo que nos
fará dormir. Aconteceu em toda a floresta. Ele procurou sem desistir,
procurou, procurou e acabou encontrando essa coisa perto da sua moradia.
A cauda da coisa já estava visível, pendurada em um galho, mas Horonamɨ
pensava que a coisa estaria sentada na raiz de uma árvore e continuou
procurando longe, em todas as direções. 

Não foi a noite que surgiu sozinha, de repente, para nós dormirmos.
Assim, quem fez não foi outro. Não foi outro que fez anoitecer: foi
Horonamɨ, e apenas Horonamɨ, quem soprou nosso sono --- somente ele. 

Qual a razão dessa procura? Como de dia ninguém parava de fazer sexo ---
vocês também não fazem sexo de dia? --- e como a noite não existia ---
era sempre luz forte do dia --- para ele esquecer os outros fazendo
sexo, ele procurou a noite para envolver todos na escuridão. 

A noite estava empoleirada em cima de uma árvore não muito distante.
Parecia com um mutum empoleirado, cuja cauda repousava na parte alta de
um galho inclinado de uma árvore \textit{paikawa}.\footnote{Árvore baixa, chamada localmente de pé-de-maçarico.}  Assim era a escuridão. Apesar de a noite parecer um mutum, Horonamɨ conseguiu
encontrá-la. A noite também cantava como um mutum. 

Nessa época, os animais --- como arara, mutum, queixada, anta, veado,
caiarara, maitaca, irara, tamanduá-bandeira, papagaio e jabuti --- eram
Yanomami e, como os Yanomami, moravam em xapono. Horonamɨ designou cada
espécie de animal e deu-lhes seus nomes. Naquela época, ele procurou
pela terra firme sem descanso, quando não havia xaponos espalhados pela
selva; havia somente o xapono dele.\footnote{  Horonamɨ realiza diversas buscas 
para encontrar tudo que os Yanomami usam para viver.}  Os animais também
viviam em xapono.\footnote{Isto é, eram gente.} 

Quando Horonamɨ soprou a escuridão com sua zarabatana para nós
dormirmos, ele queria que anoitecesse. Ele encontrou a escuridão e
soprou. Depois de fazer cair a escuridão, ao mesmo tempo se desenhou um
pequeno círculo no chão, embaixo do lugar onde estava empoleirado o dono
da escuridão.

O pai do cunhado de Horonamɨ se chamava Manawë. Ele era uma boa pessoa,
e avisou: 

 --- Ele vai achar agora! Tomem cuidado! --- avisou Manawë no xapono. 

Quando Horonamɨ flechou o mutum da noite, apesar de estar perto da sua
moradia e de retornar correndo, ele também sofreu, porque anoiteceu
de uma vez. Depois de ter soprado a noite em todos os cantos, e de ter
corrido, ele adormeceu. Naquela noite, os Yanomami também sofreram. Não
anoiteceu devagar. Até Horonamɨ passou fome, pois não tinha como fazer
fogo. Ele acabou ficando na escuridão, apesar de estar perto do seu
xapono. Como foi assim que aconteceu, a mãe dele também sofreu, todos
ficaram tontos de fome à noite. A escuridão perseguiu Horonamɨ bem de
perto, e ele estava com fome. 

Depois de a noite apagar o dia, os que moravam com ele morreram de
fome, pois comiam somente terra, comiam terra vorazmente e sofriam. Não
sobreviveram. Até seu próprio cunhado sofreu e quase morreu. Horonamɨ
ficou angustiado.

Havia então três pajés: o avô, o avô mais novo e o cunhado, e eles
esquartejaram a noite, fazendo reaparecer a luz do dia. 

Para as pessoas não comerem mais terra, Horonamɨ foi caçar. Ele nos
ensinou a caçar. Ele tinha uma zarabatana, que alguns Yanomami usam para
soprar, era isso que ele usava. Ele soprava os animais, tinha um sopro
forte, e foi assim que ele nos ensinou a matar a caça com veneno. 

 É assim, é a própria história dos antepassados. É a história
daquele que se apossou da floresta, é o início de tudo, a história do
primeiro dono da floresta, Horonamɨ.

\chapter{Ruwëri}

\letra{P}{ëma} kɨ miopë, pëma kɨ pehi taei ha, të tama. Ɨhɨ të rë tare, exi të ha
të taema? Pëma kɨ rë hɨtɨtɨwë rë miore, të taprapë. Komikomi të urihi ha
e kuopë, a taa he yatirarepë, a taema. A tapraɨ he yatiopë, kama yahipɨ
ahete ha, ɨ̃hɨ të texinakɨ pata hãpraa waikiama kupiyei ha. 

--- Kihamɨ hii hi nasiki ha pei të pata roa --- a puhi ha kunɨ, a taema, a
taei payëkou piyëkoma. 

Kama titititi a ha kuxëprarunɨ, a ha harɨnɨ, pëma kɨ miopë mai! Kama
titititi a xomi ha pëtarunɨ, pëma kɨ mio pehi mai! Ɨnaha a taprarema, ai
tënɨ mai! Titititi a rë kuprouwei, ai tënɨ a tapranomi, Horonamɨ a
yainɨ. Ɨhɨ xĩro. Horonamɨnɨ kama pëma kɨ maharixipɨ pehi rë horakenowei
Horonamɨ a yaia totihia. Ɨhɨ a xĩro yaia. 

Heao ha të pë na ha wayotinɨ, heao ha wama kɨ na wa rë wayouwei, hei të
titititi kuprou mao tëhë, mɨ haru a xĩro hiakawë kuotii kutaenɨ, ɨ̃hɨ të
nohi mohotipropë, titititi a taema. Të ka kahupropë. 

Hei ai a hikari rë prare naha, kihi Ruwëri a paa, hei a pata paoma,
paruri kurenaha a pata paoma. Paikawa kohi pata ora hitoteopë ha, të
texinakɨ pata hãpraoma. Ɨnaha Ruwëri a kuoma. Ɨhɨ Ruwëri a rë kui,
paruri kurenaha a kuoma makui, yakumɨ a he haa he yatirema. Kama
titititi a makui, paruri kurenaha a ɨ̃kɨma, mɨa kurenaha, mɨa ɨ̃kɨɨ
kuaama. 

Ɨhɨ tëhë, yakumɨ yaro, ara, paruri, warë, xama, haya, hoaxi, ãrima,
hoari, tëpë, werehi, totori, Yanomamɨ hei kurenaha, të pë hiraoma. Ɨhɨnɨ
yaro pë wãha hiraapotayoma. Kamiyë pëma kɨnɨ, pëma pë wãha yuapë. Ɨhɨ të
mɨ wakaraxi xĩro hamɨ a taeotima, taeotima, taeotima…Ai yahi ai,
ai yahi, xapono kurenaha kuo tëhë mai! Yami a përɨoma. Hei a xapono rë
kurenaha hapa pë kuoma. 

Ɨhɨ tëhë Horonamɨnɨ Ruwëri a rë horaprare, pëma kɨ rë miowei, të mɨ titi
titimaɨ puhiopë yaro, a horama, titititi kamanɨ a horaprarema. A ha
kemarɨnɨ, ɨ̃hɨ të xĩro ha a rë kemare të ha, ĩsitoripɨ komorewë titititi
a praoma. Titititi a praoma, ɨ̃hɨ a pepi ha. 

Pe heri hɨɨpɨ rë kuonowei, ɨ̃hɨ pë hɨɨ Manawë e wãha kuoma. E wãha
wãritio taonomi. Pe heri hɨɨpɨ wãha kuoma. 

--- Kuikë a tapraɨ kure. Pei pë ta moyawëpo! --- e kuu heama. 

Kutaenɨ a rë niarahari, kama a wãisipɨ ahetea makure, a rërëimama makui,
a no preaama. Rope të mɨ titirayou yaro. A ha horararɨnɨ, a rërëatii
makui, hei a mio kure. Ɨhɨ të titi hamɨ, pë no Yanomamɨ preaaɨ xoaopë.
Opisi titi a kuaaɨ taonomi. Ɨhɨ tëhë kama a makui, a no preaama, ohiri,
pohoro hi kɨ poimi yaro. Kama a ruwëmoma, yakumɨ kama a ruwëmoma, a
hiraa ahetea makure, a ruwëmoma. Ɨnaha të kuprarioma kutaenɨ pë nɨɨ e no
preaama, pë ohiri wëkëkoma mɨ titi hamɨ. Ɨnaha të kua. Kihi raxa si kɨ
rë kurenaha, ɨ̃naha e ruwëmou kuoma, yahi ahetea makui. Ohiri. 

Horonamɨnɨ pë kãi rë përɨawei ha, pë ka rë hẽaprarɨhe, ohi a wayunɨ, pë
nomaa haikirayoma. Pë xëprarema. Hei pita a yãxaamahe, a pata
wëhërɨmamahe, horema pë rë kurenaha pë no preaama. Ɨhɨ e pë hëpronomi.
Pe heri a no premapoma. Pe heri e kãi waharoprarioma. Kama a rë kui, a
xi harihirayoma. 

Hekura ɨ̃naha të pë kua yaro, pë xɨɨ, pë xɨɨ oxe, pe heri, ɨ̃naha pë kua
yaro, ɨ̃naha, ai, ai, ai, pë hekura kua yaro. Titi a ha yakëkëpraɨ he ha
yatirohenɨ, të mɨ harumaremahe. Ihiru heinaha kuwë, huya, pë
hiakapronomi, pë nomaa haikirayoma. Pë ohitima yaro. Pruka mi titi të pë
yukemahe yaro, titi a huxomi hamɨ, pë hiakapronomi, pë ruwëri no
preaama, pë ni kãi ha maprarunɨ, ihiru rope pë nomaɨ he tiherimoma,
ɨ̃naha të pë kuaama. 

Ɨhɨ hei të rë kupraruhe hamɨ, kama a ramɨ huɨ, a ramɨ huɨ, kamiyë pëma
kɨ hiraɨ ha, yaro pë niaɨ hiraɨ ha, mokawa a poimi makure, yoroa
Yanomamɨ të pë rë horaɨwehei, ɨ̃hɨnaxomi a poma. Ɨhɨnɨ yaro pë horama,
mixiã kɨ hiakao totihioma, ɨ̃hɨ të pou yaro, të pë husunɨ, të pë ixou
hiraɨ ha, ai të ihiru imisi kãi hĩrema, të rë xëprarenowei, ɨ̃hɨ rë a rë
përɨo mɨ hetuonowei, ihirupɨ xëpraɨ hayurayoma. Pore a përɨoma, hapa
kãi, Horonamɨ payeri, ɨ̃hɨ ihirupɨ rë xëpraɨ hirare, kutaenɨ, õka të pë
ha hunɨ, të pë xëɨhe, ɨ̃hɨnɨ të pë horaɨ hirama. Ɨnaha të kuwë, pata të ã
yai. Ɨhɨ urihi a rë ponowei të ã, të komosi rë praikuhe hamɨ të ã.

\chapter{Horonamɨ}
 
%\section{Quem nos fez?}

\letra{E}{sta} é a verdadeira história de nosso surgimento: quando a floresta era
virgem, apareceu Horonamɨ, personagem principal de nossa história, por
causa de seus ensinamentos. O grande pajé\footnote{Ser pajé, nestas histórias, quer dizer que o personagem em questão é ou tem a capacidade de se transformar em espírito e, com isso, fazer coisas
extraordinárias.}  yanomami Horonamɨ surgiu dele mesmo; surgiu ao mesmo tempo que esta floresta e
foi quem ensinou os Yanomami a morar nela. Assim foi o início. 

Não existia Yanomami como os de hoje, nem outro ser humano. 

Ele propagou sua sabedoria para que nossa história fosse sempre lembrada
e discutida, como fazemos agora. Aconteceu bem antes de os tuxauas
yanomami passarem a existir como existem
hoje.\footnote{No Amazonas, onde vivem as comunidades de Ajuricaba e Komixipɨwei,
usa-se \textit{tuxaua} ou \textit{liderança} para designar a pessoa de referência
de uma comunidade indígena, por essa razão optou-se por esses termos na
tradução.}  Horonamɨ foi o primeiro habitante da floresta e
nos ensinou a morar nela, assim como ensinou também aos estrangeiros,
os \textit{napë}.\footnote{O termo \textit{napë} designa os estrangeiros, em geral os brancos, ou quem adotou seus costumes.} Ele não tinha pai, mas mesmo assim
ele surgiu. Ele surgiu em uma floresta maravilhosa. 

Quem morava com Horonamɨ? Horonamɨ morava com seu cunhado, Wɨyanawë,
que, apesar de não ter desposado sua irmã, era seu verdadeiro
cunhado.\footnote{Os Yanomami, tradicionalmente, não podem chamar uns aos outros por seus nomes próprios, por isso usam termos de parentesco. Quando não há
consanguinidade, são usados termos de afinidade, como cunhado ou sogro.
Cunhado é também um termo positivo, na medida em que indica alguém em
quem se pode confiar.} Horonamɨ sempre o levava consigo nos
períodos que passavam dentro da mata, chamados \textit{wayumɨ}, e ensinou
os descendentes como ir de \textit{wayumɨ}.\footnote{Longas estadias coletivas na floresta. Em geral são motivadas pela falta de comida no xapono. A comunidade pode se dividir em vários grupos quando se trata de um xapono populoso, e se desloca num vasto círculo, fazendo acampamentos sucessivos.}

Apesar de sua mãe não ter parido Horonamɨ, pois ele surgiu de repente, o
nome de sua mãe era Yotoama. O pajé Horonamɨ foi quem procurou e
descobriu nossa comida, nosso conhecimento da floresta e o habitat dos
animais, para que, quando os Yanomami ocupassem a floresta, eles fossem
capazes de aplacar sua fome de carne. 

Ele descobriu o nome dos animais quando eles viviam como nós. Apesar de
serem animais, antes eles viviam do mesmo modo que os Yanomami. 

Como ele fez aparecer a água para acalmar a sede dos Yanomami? Ele abriu
várias veredas na floresta. Abriu veredas em todas as direções, de forma
que elas nunca sumam e que sempre bebamos água. 

Horonamɨ tinha seu próprio xapono,\footnote{Os xaponos são as casas 
coletivas circulares onde moram os Yanomami. Cada casa corresponde 
a uma comunidade; em geral não se fazem duas casas numa mesma localidade.} onde moravam
também seus aliados, que se tornaram muito importantes. 

Como se chamava o xapono pertencente a Horonamɨ? Esse xapono chamava-se
Horona. 

O xapono vizinho, que ficava do outro lado do rio, se chamava
Menawakoari. Os primeiros habitantes desse xapono também se
chamavam Menawakoari. Penewakoari era o tuxaua e morava com o grupo dos
Kapurawëteri. O tuxaua dos que moravam com Horonamɨ se chamava
Penewakoari. Kapurawë era o nome do xapono e da região dos
Kapurawëteri.\footnote{\textit{Habitantes}: em alguns casos o xapono tem o nome de seu tuxaua.}

Penewakoari morava com eles e estava destinado a se transformar num
monstro. Penewakoari depois se transformou no monstro Xõewëhena, faminto
de carne e comedor de crianças. Mas, quando ainda era Yanomami,
Penewakoari morava no xapono Kapurawëteri, vizinho ao
xapono Horona.

Nesses xaponos moravam poucas pessoas. Com o tempo, nos xaponos vizinhos
foram aparecendo mais tuxauas. Os primeiros tuxauas que viviam nos
xaponos vizinhos, os xaponos dos aliados, não eram nossos
antepassados, eram outros. Sobre eles se contaram estas histórias.

\chapter{Horonamɨ}

\letra{Y}{anomamɨ} hekura kama xoati a pëtarioma, urihi hamɨ he usukuwë a rë
pëtarionowei, Yanomamɨ përɨaɨ hirarewë a rë pëtarionowei a yai. Ɨnaha të
kua, hapa. 

Yanomamɨ hei kurenaha pë kuo mao tëhë, ai të kuonomi. 

Wetinɨ pëma kɨ taprarema? Kamiyë pëma kɨ rë pëtarionowei të ã yai kua.
Pëma kɨ rë hiranowei kurenaha pëma kɨ noã tayopë. Urihi a xomao tëhë,
Horonamɨnɨ Yanomamɨ të rë hiranowei, ɨ̃hɨ a xĩro perɨamɨ pëtarioma.
Horonamɨ Yanomamɨ të pë ihirupɨ përɨamɨ kuo mao tëhë, Horonamɨ hapa kama
hekura a pëtarioma. Pëtarunɨ, urihi a yurema. Ɨnaha kamiyë pëma kɨ no
patapɨ yai wãha kua. 

A përɨkema. Kamiyë pëma kɨ përɨaɨ hirapë. Napë pë makui, pë përɨaɨ
hirapë, hirama. Horonamɨ ai pë nɨɨ e kuonomi makui, kama a pëtarioma.
Urihi hei a kuonomi, urihi katehe a ha a pëtarioma, katehe urihi a ha. 

Horonamɨ weti xo kɨ përɨpɨoma? Kama Horonamɨ, pe heri xo, Horonamɨ pe
heri a rë pararuponowei, notiwa të kɨ wayumɨ përɨaɨ hiraɨ ha a rë
pararuponowei, pe heri Wɨyanawë e wãha kuoma, ɨ̃hɨ Horonamɨ pe heri yai,
yaɨpɨ e poimi makure, Wɨyanawë pe heri e kuoma. 

Pë nɨɨnɨ a kepranomi makui, e xomi pëtarioma, pë nɨɨ Yotoama e wãha
kuoma. Horonamɨ kahikɨ rë nɨɨmonowei kama xoati Yotoama e wãha kuoma,
Horonamɨ nɨɨpɨ. Yanomamɨ pë rarou mao tëhë, ɨ̃hɨ a rë përɨkenowei, hapa a
wãha koro prao kure. Ɨhɨnɨ Horonamɨnɨ hekura a rë pëtarionowei, kama
xoati a rë pëtarionowei, ɨ̃hɨnɨ kamiyë pëma kɨ rë iaɨwei, a urihi rë
mɨnowei, yaro pë rë përɨhɨmonowei, pë rë wãrinowei, kamiyë pëma kɨ naiki
waopë. 

Hei kurenaha kuwë të pë përɨhɨmoma, ɨ̃hɨnɨ yaro pë wãha wãrima, ɨ̃hɨ a
mori kua yaro. Yarori pë makui, e pë Yanomamɨ përɨaɨ ha parɨikunɨ,
Horonamɨnɨ pë wãrii piyëkoma. 

Ɨhɨnɨ pë amixi kãi rë kõamanowei, ɨ̃hɨnɨ wetinɨ, weti naha u pë kupropë?
Horonamɨnɨ urihi hamɨ pei yo pë reiki rë tanowei, exi të pë kupropë mai!
Mayo kɨ maprou pëo rë mai, yo kɨ tama. Mau pëma u pë koapë. U pë
kupropë, yo pë tama. Ɨnaha a urihi komio tëhë, ɨ̃naha të tama. 

Horonamɨ xaponopɨ kuoma, pë rë përɨonowei. Ɨhɨ payeri a rë
payeriponowehei, përɨamɨ të pë kuprarioma. 

Kama xaponopɨ ipa kurenaha, pukatu hamɨ, ai xapono, a rë kuonowei,
Horonamɨ kama xapono e rë ponowei, weti naha e wãha kuoma? A kãi rë
përɨonowei, nahi rë ĩtaponowei, kama Horona xapono e wãha kuoma. Ɨhɨ
kamanɨ a wãha rë yehipore a kãi përɨoma. Ɨhɨ e wãha kuoma, xapono. 

Kama kɨpɨ rë përɨpɨonowei, ɨhɨ te he tikë ha, Menawakoari a kuoma. Hapa
të pë rë përɨonowei të pë wãha, Menawakoari, Penewakoari hapa Xõewëhena
yai, yai të rë kuprarionowei, Penewakoari a naikia rë përɨonowei,
Penewakoari përɨamɨ kë a, Kapurawëteri pë kãi përɨoma. Hei Horonamɨ kama
teri e pë kãi rë përɨonowei Penewakoari përɨamɨ a wãha yai kuoma. Kama e
pë rë kui Kapurawëteri e pë wãha kuoma. Kama yahipɨ, urihipɨ
Kapurawëteri e wãha kuoma. Kama përɨamɨ Penewakoari, yai të kupropë
makui, pë kãi përɨoma. Ihiru pë wama, hei kurenaha pë wama. Hapa a
yanomamɨo tëhë, Penewakoari a përɨkema. 

Ɨnaha houkutawë, kuwëtatawë pë kãi përɨoma. Hãɨkɨtawë pë kãi përɨoma.
Ɨhɨ kama e pë rë kui Kapurawëteri. Kama përɨamɨ Penewakoari a wãha
kuoma. Ɨhɨ te he wai tikëre hamɨ, pë yahipɨ he rë tikëkëmonowei, xoati
përɨamɨ pë kuprarioma. Kama nohi pë yahipɨ he rë tikëkëmonowei, përɨamɨ
ai, hapa të pë rë kuonowei, kamiyë yama kɨ no patama mai, ai! Hapa të pë
wãha nohi rë wëyënowehei të pë wãha.
 
\chapter{O surgimento do tabaco}
 
\letra{E}{sta} é a história de Hãxoriwë, o dono do tabaco. Antes ninguém usava o
tabaco, porque ninguém conhecia suas sementes, nem as soprava para
semear. 

``É desse jeito que se coloca o tabaco no lábio!'' Ninguém pensava
assim. Eles não conheciam o tabaco; por isso, ninguém andava com
brejeira no lábio, ninguém o usava, pois o desconheciam. 

Nessa época, Hãxoriwë morava sozinho, não tinha esposa nem
filho. Quando Horonamɨ por acaso o encontrou, ele fez
perguntas a Hãxoriwë. Horonamɨ o encontrou pois era pajé e se deslocava
facilmente. Quando Horonamɨ o encontrou, ele o viu comendo a
fruta \textit{pahi}, um tipo de ingá. Hãxoriwë estava comendo, mas não
usava tabaco. Ele tinha vontade de usar tabaco, por isso chorava.
Hãxoriwë chorava. Estava sofrendo por causa do tabaco, e assim nos
ensinou a ter vontade de usar o tabaco --- por isso choramos quando não
tem tabaco. 

Horonamɨ apareceu naquele momento; Hãxoriwë estava comendo. Ele comia
frutas \textit{pahi} sem parar. Os galhos estavam cheios de frutas
agrupadas, que estavam penduradas nos galhos carregados. Horonamɨ o viu
comer. Horonamɨ estava vindo sem nada, não tinha brejeira, mas fez
aparecer no seu lábio um tabaco sem cor. Ele fez aparecer o
tabaco \textit{taratara}.\footnote{Trata-se de uma variedade forte de tabaco, muito apreciada.}  Enquanto Horonamɨ ainda estava
de pé, ele perguntou a Hãxoriwë: 

--- Quem é você? Você aí, quem é? 

--- Não pergunte quem sou! Sou Hãxoriwë! --- disse ele. --- Meu
filho,\footnote{Modo carinhoso usado por parentes mais velhos ao se dirigirem a
parentes mais novos, mais especificamente entre pais e filhos ou avós e
netos.} é você? 

--- Sim.

--- Você, quem é você?

--- Sou Horonamɨ, sou Horonamɨ --- disse. --- O que você está comendo? 

--- Não pergunte o que é! --- retrucou. --- Eu como fruta. Eu como
fruta. É a fruta \textit{pahi}! --- disse Hãxoriwë. 

Quando ele disse isso, Horonamɨ olhou. Ele queria fazer aparecer o
tabaco. Ele não fez aparecer o tabaco da forma que o conhecemos, pois
ninguém, sequer ele mesmo, sabia preparar o tabaco depois de soprar as
sementes e de misturar as folhas com cinzas. Como Horonamɨ era pajé, ele
fez sair o tabaco de dentro de Hãxoriwë. Depois de fazer sair o tabaco
sem cor, ele o usou. Hãxoriwë olhou e quando viu o tabaco: 

--- \textit{Hɨ̃ɨɨ}! --- chorou logo. 

Era um ardil para que Horonamɨ lhe desse o tabaco: 

--- Brejeira! Meu filho! Brejeira! --- chorou Hãxoriwë. 

--- \textit{Hɨ̃ɨɨ}! Meu sogro! Você está sofrendo tanto assim?! 

--- Sim! Estou querendo, meu filho! Divida o que você tem no
lábio! --- chorou ele.

--- Meu sogro está sofrendo muito, mesmo! Me dê algumas das frutas que
você está comendo e eu lhe darei tabaco para você provar! --- disse
Horonamɨ. 

Com essa conversa, Hãxoriwë jogou uma ou duas frutas. Ele estava
sovinando as frutas, guardando-as só para si. Horonamɨ experimentou as
frutas. 

Depois de chupar as frutas, os caroços caíam por si sós, de tão maduras:

``\textit{Hɨ̃ɨɨ}! \textit{Prohu}! \textit{Prohu}!'' elas faziam ao cair. 

--- Sogro! As sementes estão moles. Tem muitas frutas ali grudadas, tire
para mim! 

--- Não, primeiro me passe a brejeira! 

Hãxoriwë nos ensinou essa palavra: brejeira. Assim, quando Horonamɨ a
guardou no lábio, ele disse: 

--- Minha brejeira! 

Não apareceu logo esse nome, tabaco.\footnote{Nesta narrativa os dois termos 
são tratados como sinônimos.}  Ele só apareceu
quando Hãxoriwë pronunciou essa palavra, até então desconhecida.
Horonamɨ lhe deu a brejeira. Horonamɨ aproveitou a situação e pediu
outras frutas. Assim, Hãxoriwë lhe deu mais uma, mais uma e mais uma.
Essas frutas penduradas, depois de colhidas, pareciam cachos de banana. 

--- Vamos, meu sogro! Experimente! --- disse Horonamɨ. --- Prova!

``\textit{Tëɨ}!'', Hãxoriwë caiu. 

--- Dê aqui! Traga aqui! --- choramingou. 

Como Hãxoriwë estava chorando, Horonamɨ lhe deu o tabaco e ele
logo o colocou no lábio. Quando o colocou na boca, ele já ficou tonto, e
tremia de tontura. Ele chorava, embriagado. A força do tabaco o pegou
imediatamente. Ainda com o tabaco na boca ele cuspiu, e a espuma caiu no
chão. Onde a espuma caiu, surgiu um broto de tabaco, que logo cresceu e
se espalhou de uma vez. As folhas de tabaco logo ficaram grandes, como as folhas da jurubeba. 

Horonamɨ fez aparecer o tabaco através de Hãxoriwë. O conhecimento das sementes foi transmitido, por isso nossos antepassados as pegaram e hoje nós usamos o tabaco, apesar de ele se originar do cuspe de Hãxoriwë. 

--- Meu sogro, depois de melhorar, você dirá: é só tabaco! --- disse
Horonamɨ. 

Enquanto Hãxoriwë estava pendurado e inebriado, uma espuma grande saiu
da sua boca, por causa da força do tabaco. Ele se engasgou e cuspiu, e
foi dessa espuma que surgiu o tabaco, do cuspe de Hãxoriwë, que se tornou
tabaco. 

E um dia, quando os antepassados foram de \textit{wayumɨ}, como de
costume, um deles encontrou o tabaco. Assim, fizeram se multiplicar as
sementes e ficaram conhecendo o tabaco. 

Quem fez aparecer o tabaco? Nós já sabemos, não foi outro que o fez
aparecer. Não foi um Yanomami comum. 

Havia nessa época os Yanomami do xapono Warahiko, e foram eles
que encontraram o tabaco, foi um deles. Quando viram o tabaco,
disseram: 

--- \textit{Õooãa}! Uau! Uma plantação de tabaco! 

Foram eles que pronunciaram o nome do tabaco. Em uma região
ali perto, moravam dois Wãimaãtori, de outro xapono. Quando os do
xapono Warahiko encontraram um deles, lhe contaram a respeito do
tabaco. 

--- Meu filho! Qual é o nome disso? 
--- Ah, é tabaco! --- assim retrucaram os dois Wãimaãtori. 

Foi assim que aconteceu: Hãxoriwë, os Warahikoteri e os dois Wãimaãtori
descobriram o tabaco primeiro. Foi assim que o uso do tabaco se
desenvolveu. Os \textit{napë} não fizeram surgir o tabaco depois de soprar
as sementes. Foi a partir do lugar onde surgiu o tabaco que ele se
espalhou por todo canto. Assim foi. 

Como surgiu o tabaco? Já sabemos: Hãxoriwë iniciou o processo quando
Horonamɨ fez aparecer o tabaco, enquanto Hãxoriwë estava olhando. É obra
de Horonamɨ, foi ele quem o fez surgir. Ele é um grande pajé, por isso,
o maior. 

Depois de o tabaco se espalhar, quando os Warahikoteri eram Yanomami,
eles até desmaiaram com a força do tabaco \textit{taratara}. Sofreram de
tontura. Os dois Wãimaãtori que moravam mais além, apesar de serem
resistentes ao tabaco, também desmaiaram e ficaram duros por causa da
força do tabaco \textit{taratara}. Mas depois eles melhoraram. Foi assim
que, em seguida, pegaram as sementes de tabaco e as espalharam,
fazendo-as se multiplicarem aqui. Assim foi.

Hãxoriwë morava aqui. Depois da história do sofrimento de Hãxoriwë,
surge a história do encontro de Horonamɨ com o Tatu.

\chapter{Hãxoriwë}

\letra{H}{ãxoriwë} të ã. Ɨnaha të kua. Pẽe nahe mo ha horarɨhenɨ, pẽe nahe mo kɨ
ha tararɨhenɨ, ha horarɨhenɨ, nahe mo ha homorɨnɨ, të pë kareanomihe,
hapa. Ɨnaha pẽe nahe kareamou: 

Hata kure! Të pë puhi kunomi. Xĩro të pë puhi mohoti kuotima, ɨ̃hɨ të pë
husi kãi karereapraronomi, ai të kareanomihe, të pë puhi mohoti yaro. 

Ɨhɨ tëhë, Hãxoriwë yami a përɨoma. Hesiopɨ mai! Hesiopɨ a kãi kuonomi,
ihirupɨ e kãi kuonomi. Ɨhɨ a he ha harënɨ, Horonamɨnɨ a he ha harënɨ, a
he harema, a he haapërema, a wãrima, ɨ̃hɨ wetinɨ e të yai taprarema.
Ɨhɨnɨ rë a he rë haarenɨ, kama hekura a yaro, hei xĩro kurenaha e
warokema makui, Hãxoriwë a iaɨ ha tararɨnɨ, pahi kɨ ha a iama. Kete,
pahi kɨ ha, xĩroxĩro pẽe nahe kareponomi. A puhi toopronomi, ɨ̃hɨ të pë
ha a ɨ̃kɨma, Hãxoriwë a ɨ̃kɨma. Ɨhɨ të pë no pëxɨrɨ ha a no preaama, hei
pëma kɨ puhi toomi hirama, pëma kɨ ma rë ɨ̃kɨɨwei, ɨ̃hɨ tëhë Horonamɨ e
pëtarioma. Hãxoriwë a iama. Pahi kɨ ha a iatima. Pei hi poko kɨ hamɨ, e
të pë pata yërëkëmoma, ximokore e të pë pata reikipramoma. A iaɨ
tararema. Ɨhɨ ei të rë pëtamare, xĩroxĩro a huimama, ai e të kareponomi,
axiaxi e të pëtamarema, pei husi hamɨ. Ɨha e të rë pëtamare, taratara e
tapraɨ kure. Horonamɨ e upratou tëhë: 

--- Weti kë wa? Mihi weti kë wa? --- e kuma. 

--- Wetima! Hãxoriwë kë ya! --- e kuma --- Xei! Kahë rë wa? 

--- Awei. 

--- Weti kë wa?

--- Horonamɨ kë ya, Horonamɨ kë ya! --- e kuma --- Exi wa të kɨ waɨ kure?
--- e kuma -- 

--- Exima! --- e kuɨ no mɨhɨoma --- Kete ya kɨ waɨ, kete ya kɨ waɨ. Pahi kë
kɨ! -- e kuma. 

Ɨhɨ e mamo xatiprakema. Pẽe e nahe pëtamaɨ puhiopë yaro. Ai tënɨ, kamanɨ
të mo kɨ ha horakɨnɨ, të ha yaarɨnɨ, e të rɨpɨ pëtamanomi. Kama hekura a
yaro, pei huxomi hamɨ e hamarema. E ha hamarɨnɨ, e të karetarema axi.
Kihi mamo xatiprakema. Pẽe nahe ha tararɨnɨ: 

--- Hɨ̃ɨɨ! --- ɨ̃harë e ɨ̃kɨa xoarayoma, pẽe nahe ha, e të hipëamaɨ puhiopë
yaro, nomohori. 

--- Weyuyë këëëë! Xei! Weyuyë këëëaaa! --- e kuma. E mɨa kuma --- Hɨ̃ɨɨ!
Xoape wa puhi too no preomi totihiwë tawë?

--- Awei ya puhi tooma, xei, mihi wa të wai rë karepore! Të ta karoa haɨpa!
--- a ɨ̃kɨranɨ e kuma. E kuɨ ha: 

--- Xoayë të ã no preo rë totihiwë yai ta këɨ̃ɨɨ. Mihi wa të kɨ rë ware,
ɨ̃naha të kɨ ta hukëa tapa! Ɨhɨ hei ya të hipëapë, wa të mɨpë! --- e
kuma. 

Ma kuɨ tëhë, porakapɨ e të kɨ, mahu të kɨ xëyëkema, të kɨ no xi ɨmapou
yaro. Të kɨ nowamama. E ha xëyëkɨnɨ, e wapama. 

--- Hɨ̃ɨɨ! Prohu! Prohu! --- kama e mo kɨ prërëɨ rëoma, hĩ horehewë të kɨ
pata. 

--- Xoape, të mo kɨ pata prore totihiwë kë! Mihi xĩtoxĩto të kɨ pata rë
tëre, ɨ̃hɨ të kɨ pata ta hukëpa!

--- Ma, weyuyë a wai ta hio parɨo! --- Kama Hãxoriwënɨ weyu a wãha hirama.
Ɨhɨ kutaenɨ, a karepou ha:

--- Weyuyë kë! --- e kuma. 

Hapa pẽe nahe wãha kuo haɨonomi. E ha kunɨ, e të hipëkema. Ai kɨ ha
nomohori nakaa kõrënɨ, ɨnaha, hei ai a, ai a, ai a, ɨ̃naha e kɨ takema.
Ɨhɨ kɨ rë yërëkëawei, e kɨ pata ha hoyorënɨ, hawë kurata e të kɨ hamo
pata rii kuwë. 

--- Pei! Xoape! Hei! Të ta wapa! --- e kuma. Wapëpraa, ɨ̃hɨ rë! 

--- Tëɨ! --- e kerayoma. Hëyëmɨ kë! Hëyëmɨ kë! --- e mɨa kuma. 

E ɨ̃kɨranɨ, e hipëkema. E karetaɨ xoarayoma. Ɨhɨ ei e rë karerehe hamɨ,
Hãxoriwë a rë kui, a haɨrema. A yatiyatia haɨrayoma. A porepɨ ɨ̃kɨma,
yëtu a haɨrema, ɨ̃hɨ të ma karepore makui, kihamɨ pei kahi u pë pata
porepɨ rë prarɨrouwei, kahi u pë moxi, kuaama makui, ɨ̃hamɨ rë nahe pẽe
rë pëtore, kihi nahe pata rë homorɨhe, ɨhɨ nahe pata pëprarioma, pẽe.
Hawë kuma masi mohe pata rë yoarɨhe. 

Ɨhɨ Hãxoriwë iha nahe pẽe rë pëtamarenowei, nahe mo kɨ piyëaɨ ha
kuikuhenɨ, pëma të pë hore kareaɨ kure, pei kahi u pë makui. 

--- Xoape wa ha harorɨnɨ: \textit{pẽe nahe} wa kupë tao --- e kuhërɨma. 

A porepɨ rukëo tëhë. Pei kanehẽro pënɨ a xoaprarioma, pẽe nahe wayunɨ.
Ɨhɨ iha pẽe nahe kɨ harayoma, pei kahi u pë pẽenaheprarioma. Pata pë huɨ
ha kuikunɨ, të pë wayumɨ ma rë huɨwei, të pë ma rë përɨaɨwei, të pë
përɨama, pẽe nahe he pata rë haaɨwei, a hurayoma. 

Të mo kɨ paramaɨ xoao hërɨɨpehe, te he pata haremahe. 

Wetinɨ të kɨ pëtamarema? Pë puhi kuɨ mai! Ai tënɨ pẽe nahe pëtamaɨ
taonomi, Yanomamɨ tënɨ mai! Warahikoriteri pë hiraoma. Kama pë xĩro
hiraoma. Ɨhɨ pënɨ pẽe nahe he haremahe. Warahikoteri anɨ. Ɨhɨ pënɨ pẽe
nahe ha tararɨhenɨ: 

-Õooãa! Pẽe rë nahe pata! 

Ɨhɨ rë pënɨ nahe wãha yupraremahe. Ɨhɨ të he tikëa ha, Wãimaãtori kɨ
përɨpɨoma. Ɨhɨ Warahikoriteri pënɨ a he ha harehenɨ, Wãimaãtoriwë kɨpɨ
iha pë ã no wëa piyëkema. Ɨhɨ kɨpɨnɨ:

--- Xei! Weti naha, exi të pë wãha? Puhi ku tihehë! Pẽe kë nahe! ---
Wãimaãtori kɨpɨ kupɨma.

Ɨhɨ pënɨ, hei Hãxoriwë, Warahikoteri, Wãimaãtori kɨpɨ ɨ̃naha pẽe nahe
kareaɨ rë xomaonowehei pë kuprarioma, te he haa rë xoamakenowehei. Ɨnaha
a kupro hërɨpë, pẽe. Napë pënɨ të mo kɨ ha horakehenɨ, napë pënɨ a kãi
tapranomihe. Taprano hei ami, napë pë iha. Ɨhɨ a urihi rë kutarenaha
nahe pëtopë ha, a xomi tapramaɨ xoarayo hërɨma. Ɨnaha a kuprarioma. 

--- Weti naha pẽe nahe kuprarioma? --- puhi kuɨ mai! Haxõriwënɨ. Horonamɨnɨ
e nahe hipëkema. Kama hëyëmɨ e nahe pëtamarema, kama mamo yëo tëhë. Ɨhɨ
unosi yai, Horonamɨnɨ të rë pëtamarenowei, të yai. Kama hekura a yai
pata, pë hɨɨ a yaro. Pë hɨɨ yai. 

Hei pë rë kui, ei a rë piyërëahei, ɨ̃hɨ Warahikoteri pë rë kui, pë
Yanomamɨ kuo tëhë, hei pẽe nahenɨ, taratara a wayunɨ pë nomarayoma. Pë
porepɨ no preaama. Hei kɨ he rë torepɨre kɨ no motahapɨwë makui hei
taratara anɨ, kɨ kãi nomawë kaxexëpɨwë no prepɨoma. Ɨhɨ makui, waiha
kɨpɨ haropɨrayoma. Kutaenɨ hëyëha nahe mo kɨ piyëremahe, piyëa
xoaremahe. Nahe mo kɨ piyëaɨhe, hëyëha a raroa piyëkema. Praukou xoaoma.
Ɨnaha a kuprarioma. 

Hei Hãxoriwë a përɨoma, hëyëha. Ɨhɨ të mɨ amo ha, hei a no rë preaamare,
hei a no rë premarɨhe, a ha hayuikunɨ, Mororiwë a he hõra haa piyërema.
 
\chapter{Horonamɨ e o tatu: O surgimento do cipó e da embira}

\letra{O}{Tatu} era Yanomami e era muito comprido.\footnote{Era gente, e tinha os hábitos e o corpo semelhantes aos dos Yanomami. Trata-se aqui do tatu-de-rabo-mole-comum (\textit{Cabassous unicinctus}).}  Horonamɨ encontrou o Tatu. Por que Horonamɨ cortou o Tatu bem na cintura? Nós, Yanomami, amarramos terçados e fazemos as cordas de arco com o cipó-de-apuí que se ergue na
mata. Nós o cortamos e descascamos. É com isso que nós amarramos nossas
redes, com as embiras de cipó-de-apuí. 

Horonamɨ cortou o Tatu. Antes disso não havia linha de pesca. Nossos
antepassados não tinham corda de rede. Depois de encontrar o Tatu,
depois de esticar suas tripas, depois de destruí-lo, ele o cortou em
pedaços. 

Foi Tatu quem fez aparecer o machado, pois foi ele quem o fabricou. Ele
percebeu que certo tipo de madeira dura parecia um cabo de machado.
Assim, o Tatu possuía o único machado. Ele ensinou aos \textit{napë} como
fabricar o machado. Então ele não tinha dificuldade em tirar o mel, pois
tinha o machado. Ele fez um cabo comprido, depois de quebrar um pau,
enfiou e amarrou o machado de pedra em um pau, era um machado de pedra;
depois de amarrá-lo, ele partiu um tronco e tomou mel. Os antepassados
não tomavam mel, não sabiam tomar. Ele ensinou a tomar mel, ele que
existiu primeiro, quando os Yanomami não existiam, quando este inventor
não morava entre eles, ele ensinou a tomar mel. Esse tatu se
chama \textit{moro}. Horonamɨ o encontrou. 

\textit{Ku, kõu, kõu, kõu, kõu, kõu}!, fazia Tatu, cortando o tronco.
Horonamɨ ouviu esse som pela manhã. 

--- \textit{Ho}! Quem produz esse som, eu quero ver. Dá para ouvir de longe --- disse
Horonamɨ. 

Ele logo foi em direção ao som. O Tatu estava sozinho; o som fazia
zoada. Horonamɨ estava indo na direção do som e parou. Tatu derramava o
mel \textit{tima},\footnote{Mel de uma abelha de mesmo nome, que faz sua 
colmeia no oco dos troncos, próximo ao solo.} ele o derramava de uma árvore à qual deu o nome
de \textit{roa}.\footnote{Árvore alta e de madeira dura.} Horonamɨ ficou de pé parado, perto de Tatu, fazendo um som com a boca para chamar sua atenção. Aí fez
outro som com a boca, mas Tatu nem olhava, ele cortava sem parar, com as
pernas abertas. Naquela época, ninguém chamava o outro de \textit{sogro}.
Horonamɨ nos ensinou então a chamar de \textit{sogro}:\footnote{Sogro, ou tio. O uso desse termo indica uma relação de respeito. Horonamɨ quer se aproximar de Tatu. Trata-se também de uma observação irônica, pois as mulheres ainda não existem no período em que acontecem as histórias de Horonamɨ, e portanto as relações de aliança --- sogro/\,cunhado --- não são uma possibilidade.}

--- \textit{Hɨ̃ɨɨ}, meu sogro! --- disse. --- Meu sogro! --- disse Horonamɨ com
uma voz assustadora. 

Quando disse isso, o Tatu parou. 

--- \textit{Ɨ̃ɨ̃}! \textit{Õ}! --- disse assustado. --- \textit{Ɨ̃}! \textit{Õ}! De quem é 
essa voz? --- O Tatu falava assim. --- De quem é essa voz? --- ele respondeu, com uma voz que
não era normal. Era o seu jeito de falar mesmo. 

Horonamɨ olhou, sorriu.

--- Sogro! O que você está comendo? O que é isso? --- disse Horonamɨ. 

--- Não pergunte quem eu sou! --- ele disse. --- Você sabe quem eu sou!
Sou o Tatu! --- disse ele. Dizendo isso, ele perguntou: 

--- Qual é o seu nome? --- ele desafiou Horonamɨ a dizer seu nome. 

--- \textit{Ɨ̃ɨ}, eu sou Horonamɨ. 

Horonamɨ falava com uma voz bem bonita, pois ele era bonito. 

--- \textit{Hɨ̃ɨ}, meu filho, eu sou o Tatu. 

O Tatu era esbranquiçado. Ele era branco, como os \textit{napë}. Ele o
chamou logo. 

--- O que você está querendo fazer? O que você está cortando? 

--- Ɨ̃ɨ̃! Estou comendo assim! Estou comendo isto.

--- Eu quero experimentar --- disse Horonamɨ. --- Quero experimentar um
pouco! Posso beber? Que tipo de mel é? 

--- Não pergunte o que é! É o mel \textit{tima} --- disse o Tatu. 

A partir desse momento, nós, Yanomami, aprendemos a chamar esse mel
de \textit{tima}. 

--- Lá tem mel \textit{tima}! --- ao vê-lo, eu direi assim. 

Foi o Tatu que ensinou o nome. Horonamɨ chegou mais perto daquele que
estava falando. O Tatu maroto chamou Horonamɨ. 

--- Vai! Experimente, meu filho! Experimente, meu filho! O buraco da
colmeia ficou aberto. Pise nesse buraco e entre nela! --- disse. 

Era uma armadilha para fazer Horonamɨ entrar no buraco da árvore. Horonamɨ
aceitou: 

--- \textit{Hɨ̃ɨɨ}! Será que o buraco tem espaço suficiente? O mel está jorrando,
está gotejando mesmo. O buraco da colmeia está em baixo. A colmeia acaba
aí. Entre lá dentro! Fique mais em cima, pise para baixo! Eu estou
olhando! --- disse o Tatu, malicioso. 

Quando ele disse isso, Horonamɨ cedeu e entrou logo. Foi logo e entrou,
a colmeia fazia barulho, e ele foi até o alto da colmeia. Ficou de pé lá
no alto dela. De pé, onde ele entrou, pelo buraco que o Tatu tinha feito.
O Tatu fechou o buraco, e não havia outra saída. O Tatu prendeu Horonamɨ
lá em cima. Horonamɨ gritava lá dentro. Não tinha como sair. Se Horonamɨ
fosse um Yanomami como outro qualquer, ele jamais sairia. Ele gritou e
gritou lá de dentro, sofrendo, gritando e chorando. Chorava como
criança. O Tatu, que o prendeu, fugiu correndo para longe. Aquele que
estava preso por si só fez espocar a árvore. O Tatu já estava longe. 

--- Ele não vai me seguir --- pensou o Tatu, muito seguro de si. 

Horonamɨ, com seu pensamento e seu sopro forte, arrebentou a
árvore \textit{roa}. Ele ficou de pé e olhou ao redor, mas o feioso que o
prendeu não estava mais ali. Horonamɨ ficou sozinho. 

--- \textit{Hɨ̃ɨɨ}! 

Depois de pular com a explosão, passou pegando a dala e a zarabatana que
estavam penduradas. Colocou nas costas. 

--- \textit{Hɨ̃ɨɨɨɨ}! --- gemeu. --- O que tem o nome de Moro, esse feioso,
ele ferrou comigo! --- disse, triste.

Horonamɨ não errou de lugar: ele correu logo para onde o Tatu havia ido,
e foi rápido, ensinando-nos a correr. Horonamɨ correu na direção do
lugar onde havia muitas pedras saídas da terra; ele correu e correu,
seguindo os rastros do Tatu, como fazem os cachorros. Daí, Horonamɨ
correu dando uma volta, e cortou o caminho do Tatu. Horonamɨ o
encontrou e o Tatu se assustou. Como o Tatu o havia prendido, ele ficou
com medo e com raiva por dentro, e tentou agradá-lo, mas não conseguiu
suscitar a compaixão de Horonamɨ. 

O Tatu apareceu.

--- \textit{Taha}! \textit{Arrá}! --- disse Horonamɨ. 

Era mesmo o Tatu. Ele espreitava, com a mão sobre a testa, à procura de
mel. Olhava passando entre as árvores. Horonamɨ já estava de pé, pegou
um atalho e deu uma volta. O Tatu se confundiu na floresta e acabou
chegando justo onde estava Horonamɨ. Horonamɨ estava de pé, atrás da
árvore, e deu um susto grande nele. Horonamɨ queria cortar aquele que o
havia aterrorizado. Ele decidiu levá-lo até um tronco, fingindo que ali
havia uma colmeia, para fazê-lo se abaixar. O Tatu pegou o machado. 

--- \textit{Hɨ̃}! Meu filho, aqui está! Aqui está! --- disse. --- \textit{Hõ, hõ, hõ,
hõ}! Meu filho! \textit{Hõ, hõ, hõ, hõ}! Venha cá ver! Olhe aqui! Meu filho, aqui
está! --- disse Horonamɨ. 

Horonamɨ dizia isso tentando agradar o Tatu, e ia indo atrás dele. 

--- \textit{Hɨ̃ɨɨ}! Me passa isso que você tem aí no ombro, está afiado
mesmo? --- disse Horonamɨ, astuto. 

A falsa colmeia fazia barulho, e Horonamɨ fez diminuir esse barulho,
para que o Tatu abaixasse a cabeça para ver melhor a colmeia. Enquanto o
Tatu olhava para a colmeia com a cabeça abaixada, enquanto ele estava
nessa posição baixa, ele dizia:

--- Aqui está a entrada da colmeia! 

Quando o Tatu disse isso, o machado já estava na mão de Horonamɨ e,
enquanto o Tatu abaixava a cabeça, Horonamɨ o cortou bem na cintura. 

\textit{Krihii, kriihii}!, fez Horonamɨ, cortando o Tatu para se vingar, pois
ele tinha sofrido por causa do Tatu. 

--- \textit{Ëëëëããaaë}! --- gemeu a parte de cima do longo corpo do Tatu. 

Apesar de ser só um pedaço, a parte superior correu embora, sofrendo. Do
lado de cá ficou a parte inferior; as tripas vinham se esticando e a
parte superior ficava rolando. Assim, as tripas foram se esticando até
lá, elas não se arrebentaram. A parte superior daquele que Horonamɨ
havia cortado, e que ele queria que se tornasse o tatu \textit{moro}, foi
lá para cima, até onde estão os espíritos. Foi para lá que fugiu a parte
superior do Tatu. Aqui no chão ficou a parte inferior. 

Só um pedaço do Tatu chegou aos espíritos. Suas tripas não apodreceram;
elas foram até onde se erguem as árvores e subiram nelas. Uma parte das
tripas do Tatu se transformou em cipó-de-apuí e outra parte se
transformou na embira \textit{xinakotorema}, com a qual, depois dessa
transformação, os Yanomami começaram a amarrar as cabeças das redes de
cipó. Foi assim.

Apesar de nossos antepassados saberem fazer redes de cipó, eles se
deitavam no chão, pois não havia corda. Eles se deitavam no chão ---
colocavam a rede de cipó no chão para deitar. 

Como foi que eles descobriram a rede de cipó? Eles não sabiam descascar
o cipó-titica com os dentes, então era assim.\footnote{O cipó-titica é usado na fabricação de cestos.}Até as
moças deitavam no chão. Deitavam uns em cima dos outros, como os
cachorros. Sofriam na escuridão. Eles eram assim. Dormiam passando frio.
Para que nossos antepassados não passassem mais necessidades, as tripas
de Tatu se tornaram cipó-de-apuí que amarra as redes. Foi assim. 

Depois da transformação das tripas, eles passaram a usar o cipó para
fazer terçados e machados de pedra, e para amarrar a cabeça das redes,
também feitas de um tipo de cipó. Depois, com o passar do tempo, eles
teceram cestos. No início eles também não sabiam tecer cestos. Assim
foi. Esta história acabou.

\chapter{Mororiwë}

\letra{Ɨ}{hɨ} Mororiwë Yanomamɨ a kuoma, a rapeoma. Hei a he haa piyërema,
Hãxoriwë a wapëa hayurema. 

Ɨhɨ exi të ha a rii pëprarema? Pëɨxokɨ pëpraɨ rë piyërayonowei. Yanomamɨ
pëma kɨnɨ sipara pëma pë õkapë, hãto pëma nahi tana pë tapë, xĩki pë
uprahaapë. Xĩki a kuo tëhë pëma a ha hanɨrënɨ, pëma a kãi hikekeaɨ. Ɨhɨ
anɨ pëma kɨ pëkɨ he õkaopë, xĩki pë kupropë. 

Mororiwë a pëprarema. Ihiya masitana pë kuonomi. Pëma kɨ nohi patama
pëkɨ tana pë kãi kuonomi. Ɨhɨ a he ha harënɨ, xikɨ ha hĩrihou xi ha
wãria hërɨnɨ, hemata a pëprarema yaro. 

Mororiwë hãyokoma kama e posi rë pëtarionowei, kamanɨ posi taprarema.
Himaro a ha tararɨnɨ, hãyokoma kurenaha e të kuoma. A ukërema, a ha
ukërɨnɨ ɨ̃hɨ kama Mororiwënɨ rë a hãyokoma mahu poma. Napë pë iha të taɨ
hirapë. Mororiwë a makui a xĩro no preaanomi. Napë pë iha hãyokoma a
tapramapë, a ukërema. Hawë hãyokoma a kure a hĩikema, poo e maro kuoma a
kora ha õkakɨnɨ, puu a wama. Kamiyë pëma kɨnɨ puu pëma pë wanomi. Pata
të pënɨ puu pë wanomihe, u pë kãi koaɨ taonomihe. Ɨhɨ tëhë, ɨ̃hɨnɨ puu pë
waɨ hirakema, kama a rë kuo xomaonowei, Yanomamɨ të kuo mao tëhë, të
puhi rë taowei të përɨo nikereo mao tëhë, ɨ̃hɨnɨ puu pë waɨ rë
hirakenowei kë a. Moro pë wãha kua. A he hõra harema: 

--- Kou, kou, kou, kou, kou, kou! --- e kuɨ përaoma. Harika a he hõra
harema. 

--- Ho! Weti a hõra, ya të mɨɨ ta yaio hërɨ kë? Të hõra karëhou ayaa --- a
ku hërɨma. 

Ɨhamɨ e katitia xoarayoma. A hõra morokotaa tayoa yaro, ai a payeri
kuama mai! a hõra karëhoma. E rë huimiiiiii, e uprakema. Hei a tuyëɨ.
Tima e tuyëma. Roa iha wãha tapramapë, roa hi ha a tuyëma. E upratarioma. Xoape! Ai të kãi kunomi. Ɨhɨnɨ të pë xɨɨmou hirakema. E upratarioma. E kahikɨ sukusukumorayoma. A ma tahamore, e mamo
xatipraonomi. E paxëpaxëmoma, e rerekeranɨ. 

--- Hɨ̃ɨ, xoape! --- e kuma --- Xoape! --- e kuɨ no kirihiwë pëtarioma. 

E kuɨ ha, e tɨraprakema: 

-Ɨɨ̃! Õ! --- e kuma, a ãtiprario yaro --- Ɨ! Õ! Weti kë wa wã? --- e
kuma. Ɨnaha a wã haɨ kuoma. --- Weti kë wa wã? --- e kunomi. 

Ɨhɨ ɨ̃naha kama a wã rii haɨ kuoma. E mamo xatiprakema. E kahe
watetarioma. 

--- Xoape! Exi wa të waɨ kure? Exi kë të? --- e kuma --- Õ! Weti kë wa?

--- Wetima! --- e kuma --- Wetima! Mororiwë kë ya! -e kuma --- Mororiwë kë
ya. Ai weti naha kahë wa wãha kua kure? --- e kuma, a wãha kãi
yupramarema. 

-Ɨɨɨ! kamiyë Horonamɨ ya ta kui! --- e kuma. 

A wã kãi haɨ totihitao he parooma. A riëhëwë yaro. 

--- Hɨ̃ɨ̃! Xei! Kamiyë Mororiwë kë ya! --- e kuma. 

E pruxixioma. Weti a au nikerea kure? A auoma. Napë pë au rë kurenaha. A nakaa xoarema. 

--- Weti naha wa të tapë xoapë? Wa të paxaɨ ta kurawë? 

-Ɨɨ̃! Pei ya të waɨ! Pei ya të waɨ! 

--- Ya të wapaɨ puhia ta kuranɨ --- e kuma --- Ya të wãisipɨ wapaɨ puhia ta
kuranɨ! --- e kuma --- Ya të u koapë kë! Exi naxomi kë të? --- e kuma. 

-Ɨɨ̃!, exi të ma! Tima kë a --- e kuma. 

Ɨnaha Yanomamɨ pëma kɨ kuɨ hëopë: 

--- Kiha tima a kua --- ya ha tararɨnɨ, ya kupë. 

Ɨhɨ të hirama. Të wãha yupraɨ hiraɨ ha. A rë kure e ukukema. A nomohori
nakarema. Ɨnaha të pë kuaaɨ puhio yaro. Kamiyënɨ pë nomohori ha
nakarënɨ, pë no xëa rë kurepɨwei naha, a tapraɨ puhio yaro, a nakarema: 

--- Pei! Wapëpraayo! Xei, wapëpraayo! Xei! Hei oraora u nanoka pata hëkei
kuhe! U nanoka pata ta kakukuprario, hëyëmɨ wahë kɨ ha rukëtarionɨ --- e
kuma. 

A nomohori rukëmapë. E ha kunɨ, e kuɨ ha, e no xi kãi ɨmaonomi. 

--- Hɨ̃ɨ! Wa të hi ka yawëtëa ta yairawë! Hei të u pë nia pata weoweo, të u
pë nia pata xararawë nohi yaii! Hei u pata koro, hei kë! hëyëmɨ u he
pata tatoa kure! A ta rukë taru! Kiha wahë kɨ he ha torehe tarunɨ, ɨ̃naha
u pata kakukupɨa taya hërɨ! Ya mamo yëo tëhë! --- e xomi kuma. 

A kuɨ ha, e rukërayoma. E ihetarioma. E ha ihetarunɨ, Horonamɨ e
rukërayoma. E rë kõririmo hërɨɨwei, u he pata tatoopë ha, e upra
parihirayoma. Upra paru hurunɨ, hei a rë rukëmare ha, a rë pëpramouweinɨ
ta ka komipramarema, ai e te hi ka kuonomi. Kiha a xi wãri parihirayoma.
Kohomo hamɨ a kõmɨmaɨ kupoti. A no hapimi yaro. Hei kamiyë Yanomamɨ pëma
kɨ rë kurenaha, a rukëɨ ha kunoha, a no yokëɨ kõtaopɨ rë mai! A rarɨma,
kihamɨ a wã kõhomoketayoma, a rarɨprarou no preoma, a ɨ̃kɨma. Ihiru
kurenaha a miomiopraoma. Hei a ka rë kõmapramarɨhenɨ, e tokurayoma. A ka
komaprarema yaro. Hei a xi rë wãrimakɨhe, kihamɨ e rërërayoma. A pëka rë
kahure kama pehi hõra homoprou hërayoma. Prahaa waiki tare: 

--- Ware a nosi yauaɨ mai tao! --- e puhi xomi ha kunɨ. 

Kama puhinɨ, kama mixiã kɨnɨ, roa hi pata hëtɨmarema. 

--- Hɨ̃ɨɨ! Pou! --- a upratarioma. Wãriti tënɨ a ka rë kahupraɨwei, a
mɨprarema kuonomi. Yami a hëtarioma. 

--- Hɨ̃ɨɨ! --- e kuma. 

E ha yutupraikunɨ kama ruhu e ma kɨ pesi rë rukëpouwei, ma kɨ pesi
hayurema. Të kɨ ha yehitarɨnɨ: 

--- Hɨ̃ɨɨɨ! Pei a wãha yuamou Moroa rë wãritire, a no hore huxuaɨ mata yai
tanɨɨɨɨ --- a kuɨ he yautarioma. 

Yai hamɨ e kãi huɨ mai! Ɨhɨ kama a hu hërɨpë hamɨ e rërëa xoarayoma, e
hua xoarayo hërɨma, të pë rërëaɨ hiraɨ ha. Maa pruka ma pë pata
ureremopë hamɨ, a xomi ma rërërayonowei, ɨ̃hɨ rë mayo hamɨ e rërëa
hërayoma, hiima kurenaha. Kihamɨ a rë rërëre, kihamɨ e xokei tëhë, hei
kama a rë kui, a xokei tëhë, mɨ yapaɨ tëhë, hëyëha a mɨ heturema,
Mororiwë e kirirarioma. A ka kahuprarema yaro, e kirirarioma, a asimoma,
a wã xomi hirama, a nohi no ohotaamopɨma mai! E pëtarioma 

--- Taha! --- e kuma. 

No yaipɨmi. E huko si yohoa taroma, puu na pë mɨɨ ha. Hii hi pë koro
hamɨ e kuapraroma. Hei a ma upraa waikire ha. Hëyëmɨ a he rii
tiheriprou, kihamɨ a xomi rë xokeprora kiri, a nohi rë mohotuaimi, hei
te hi kɨ mɨɨ rë katitirayoi ha, a upraoma. Hëyëmɨ a ayõriprou, a
upraoma. E mɨ yami kerayoma. ĩƗha a rë kirimare, ɨ̃ha rë a pëprarema. Hei
a rë ruruare, hëyëha e nakɨ pëka tamakema, e kuami makui, a ruruapë,
kama mohe potamapë. Hãyokoma e yurema. 

--- Hɨ̃!, xei, hei kë a, hëyëha a kua kure --- e kuma --- Hõ, hõ, hõ, hõ,
xëtëwë të wai, hõ, hõ, hõ, hëyëha kë të ta mɨpra ayo, xei, hei kë --- e
kuma. 

A xomi yokomama. Ɨhɨ a ma kuɨ tëhë, e ayõprarioma. 

--- Hɨ̃hɨ, mihi të ta hiprao! Wa të rë rukëpore të namo? Të namowë, namo kë
të! --- e kuɨ topraroma. 

Hëyëha e nakɨ makuonowei makui, e nakɨ ã mɨ wëtëa piyëmakema. E nakɨ
mɨmapë. Ɨnaha e nakɨ mɨɨ ha, e kutou tëhë: 

--- Hëyëha hei të ka wai --- e kutou tëhë, ɨ̃hɨ e rë yure, mohe potou tëhë,
pei pëɨxokɨ yai ha a pahetiprarema:

--- Krihii, kriihiii! --- kama a no yuo ha, kamanɨ a no preaamaɨ tikooma
yaro. 

-Ëëëããããë! --- oraora e kuma, a ma hematai, hëyëmɨ a no preaa hërɨma. 

Kihi korokoro a rë praa hërati hamɨ xikɨ hĩrihou hëoimama, oraora a rë
yapuro hërɨɨwei, yapuro hërɨɨ, yapuro hërɨɨ, yapuro hërɨɨ, ɨnaha xikɨ
hĩrihou kurakiri, xikɨ hëtɨnomi, kama a rë pëprarɨhe, oraora a rë kui. 

A Mororiwë praɨɨ puhiopë yaro, hekura pë ihamɨ, a ora hurayoma. Ɨhamɨ a
ora tokurayoma. Kiha korokoro a prao hëoma. Ɨhɨ Mororiwë a waropë
hemata. Ɨhɨ xikɨ rë kui, xikɨ kãi tarei maopë, hii hi kuopë hamɨ, xĩkɨ
kãi tua xoape hërɨma, xĩki a kuprarioma. Ai xikɨ xĩki kuketayoma. Kihamɨ
ai xikɨ katɨrayo hërɨma. Kihamɨ xinakotorema a kuprarioma, ɨ̃hɨ xi pë
hamɨ. Ɨnaha xi pë rii ha kurarunɨ, të pë pëkɨ he õkaoma, hapa. Ɨnaha a
taprarema. 

Hapa të pë pëkɨ tao makui, pei të pë praoma. Tona kɨ kuami yaro. Të pë
praoma, të pë përɨapë hamɨ, të pë pakohepramoma. 

Ɨhɨ weti ha pëkɨpëkɨ a ha tararɨhenɨ? Të pë mohoti yaro, hei masi pë
makui, too toto pë makui, të pë kãi waxaɨ taonomihe. Ɨnaha të kuoma.
Kuwë yaro, të pë moko makui të pë praoma. Hiima pororoo kurenaha të pë
kupramoma, të pë kuaama. Ruwëri kë të pë no preaaɨ kure. Ɨnaha të pë
kuoma. Sãihiri, të pë prapramoma, ɨ̃naha. Ɨnaha të kuoma. Kamiyë pëma kɨ
no patama hõriprou maopë ɨ̃hɨ Mororiwë xikɨ xikiprarioma. 

Ɨhɨ xikɨ ha kuprarunɨ, sipara pë, poomaro pë wai hiimamahe, të pë pëkɨ
he kãi õkaoma, të pë opi puhi ha taorɨnɨ, yorehi si pë kãi tiyëmahe, wɨɨ
pë kãi tiyëɨ taonomihe, hapa. Ɨnaha të pë kuaama. Ɨhɨ ei të ã rë kui, të
ã makema.

\chapter{O surgimento da banana}
 
\letra{A}{história} da banana-pacovã. No início era assim. Nossos antepassados
surgiram e não sabiam plantar bananas. Não fosse por isso, não haveria
essas bananeiras. Não teria aparecido esse tipo de banana. 

Como pensou e agiu aquele que fez surgir a banana, depois de morar e se
estabelecer? Geralmente a gente vai à mata e encontra um lugar como se
alguém tivesse roçado, um lugar queimado e limpo, bem no meio da selva.
A gente chama esse lugar de \textit{queimado do Fantasma}. Nesse tipo de
lugar se encontra um telhado de palha, como aquele que nós costumamos
tecer. 

Embora ninguém tenha dito ao Fantasma, ``teça as palhas assim!'', ele as
teceu, apesar de ninguém ter ensinado para ele. Depois de Horonamɨ ver o
queimado, ele encontrou o Fantasma, dono do queimado, que morava ali.
Nesse tipo de lugar, erguem-se os pés de sororoca, que são semelhantes
às bananeiras, mas não dão banana. 

O surgimento das bananeiras, não foi porque o Fantasma cortou, queimou e
roçou a sororoca. Ele não as plantou. Elas simplesmente surgiram no dia
seguinte. 

\textit{Proto}! \textit{Pauximɨ}! \textit{Proto}! \textit{Rokomɨ}! \textit{Proto}! \textit{Monarimɨ}!
\textit{Proto}! \textit{Pakatarimɨ}!
\textit{Proto}! \textit{Nakoaximɨ}! \textit{Rokoya}! \textit{Rokoroko}! \textit{Roorewë}! 

Estas bananeiras e sororocas simplesmente saíram delas mesmas. Dois dias depois, o Fantasma voltou ao lugar onde havia queimado as sororocas e viu que tinha nascido
também batata-doce. Não foi em outros xaponos que ele pegou. Lá onde
Fantasma tinha seus alimentos, onde havia as bananeiras, as sororocas se
transformaram em bananas-pacovãs e a batata-doce surgiu. Ali também dava
cará, ária, pimenta e o mamoeiro. Foi o Fantasma que fez aparecer as
bananeiras. Elas vêm do Fantasma. 

Por que ele as fez aparecer? Porque ele tinha um filho, que ele tinha de
alimentar. 

Ao ouvir a voz do filho do Fantasma, Horonamɨ descobriu a sua moradia e
pegou com ele umas mudas de bananeira. 

O Fantasma não tinha outros parentes. Ele mostrou aos Yanomami que é
possível ter somente um filho. Ele fez apenas um filho, apesar de sua
esposa ser moça. Agora ele não é mais pajé, como foi em vida. 

Aquele que vinha, Horonamɨ, encontrou as bananeiras e pediu mudas ao
Fantasma. Quando não existiam nem roças, nem Yanomami, depois de
Horonamɨ pegar as bananeiras, ao chegar ao seu xapono, ele deu nomes a
elas, deixando com isso o ensinamento de como plantar as bananeiras. Ele
as pegou para nós as termos. Até hoje existem as bananas de diferentes
variedades: \textit{rokomɨ}, \textit{nakoaximɨ}, \textit{rokoya}, \textit{pauximɨ,
monarimɨ, pakatarimɨ}. Assim foi. 

Nossos antepassados e os antepassados dos \textit{napë} não comeram banana
desde o início. Hoje, tanto os \textit{napë} quanto os Yanomami plantam
bananas, a partir do ensinamento de Horonamɨ.

\section{Como os napë descobriram a banana}

Como aconteceu a descoberta da banana pelos \textit{napë}? Qual foi o
Yanomami que levou as bananeiras aos \textit{napë}? Ninguém levou as mudas
de bananeira aos \textit{napë}. Uma moça estava reclusa.\footnote{Quando a menina yanomami tem sua primeira menstruação, ela fica em
reclusão por um período entre uma semana e dez dias, dentro de um
pequeno cômodo feito de folhas de açaí no xapono. Essa reclusão a
protege do assédio de espíritos num momento em que ela fica em
evidência. Aqui a moça atrai o interesse do rio, que a carrega para fora
do xapono para se casar com ela.} A água saiu
e as roças afundaram. Essa água levou a mulher e por onde a levou, levou
também as bananeiras afundadas, até aonde os \textit{napë} vivem; foi o
rio que levou as bananeiras para que eles, os \textit{napë}, as
descobrissem. O rio desejava a mulher menstruada porque ela era bonita.
No que ela se tornou? O rio a levou porque a desejava. Da mulher
menstruada que as águas levaram, sua imagem se espalhou nos rios.
Multiplicou-se a partir dela mesma. Foi a água que a pegou. O rio
disse: 

--- Meu sogro, quero uma mulher! Me dê a sua filha! 

O rio entrou, perseguindo a mulher. O rio entrou rápido. Olha só a água!
Ela entrava por trás das casas, apesar de a terra ser alta. 

--- \textit{Prako}! \textit{Prako}! --- dizia o grande rio. 

O pai mandou pintar a filha, nessa hora ele a pintou, seu irmão a
pintou. O pai mandou seu filho pintá-la. Ele estava com muito medo de se
afogar na água, que vinha ameaçadora, se mexendo como em plena
tempestade. A água se mexia com grandes banzeiros, nos quais a mulher pintada foi
jogada, apesar da sua beleza. Seu pai a fez afundar. O rio levou a sua filha, e não a devolveu. Ela não se afogou, e o rio a
levou como sua esposa. 

--- Eu, apesar de ser água, farei dela a mãe d'água! Eu vou pegá-la ---
disse o rio. 

Por isso, esta Yanomami se tornará a mãe do rio. O rio se retirou.
Depois de pintarem seu rosto com desenhos bonitos, colocaram penas de
cauda de papagaio nas suas orelhas. Feito isso, as folhas de açaizeiro
da reclusão foram removidas e a água entrou. O xapono dele era como os
nossos. 

--- Mãe! Mãe! Pinte minha irmã! Enfeite-a! Enfeite-a depressa! --- disse
o irmão da moça.\footnote{A moça enfeitada normalmente seria entregue a um marido humano, não a um
marido rio.}

--- Essa ideia dói muito, meu filho, mas não tem jeito, entregue mesmo
tua irmã! 

Apesar de ser o rio, assim falou o pai. Ele mandou entregar a filha.
Foi assim que ele disse. Existe um canto sobre a mulher levada pelo rio,
há um canto sobre ela:

\begin{verse}
\textit{Xiri tõi!\\
Xiri tõi,\\
Xiri tõiwë,\\
Xiri tõi,\\
Xiri tõi,\\
Xiri tõi,\\
Xiri tõiwë!}
\end{verse}

Ela cantou. Quando ela pronunciou o nome de seu marido, o rio
respondeu: 

--- \textit{Tuuuuuuuuuuuu}!

--- \textit{Xiri tõi! Xiri tõi! Xiri tõi!} --- cantou o pai. 

Ele falou assim, cantou assim e, quando parou de cantar, o xapono quase
caiu, levado pelo rio. O irmão a pegou para jogá-la, apesar de ela estar
chorando. Ela chorava, por causa do seu irmão: 

--- \textit{Ɨ̃ɨaaaɨ̃ɨ}! Meu irmão! Meu irmão! Não fique triste! Meu pai! Meu
pai! Não fique triste! Minha mãe! Minha mãe! Não fique triste! 

Enquanto ela chorava assim, o irmão a pegou. 

--- \textit{Hɨ̃ɨ}! --- \textit{Kopou}!, ele a jogou de cabeça. 

Fazendo assim, a água a pegou e logo a levou. O rio cheio já estava
esperando. Quando o rio se retirou, revelou uma grande extensão de
terra. 

--- \textit{Puuu}! --- disse o rio. 

Foi assim, o rio desceu de uma vez só. 

--- \textit{Aëëë}! --- ela disse. 

A mulher se tornou boto, aquele que boia na superfície da água, pois a
jogaram na água quando ela estava menstruada; ela estava de reclusão, a
vagina dela estava ainda sangrando. Por isso se tornou a mãe d'água. 
A imagem dela se espalhou e ocupou todos os rios. Aquelas
bananeiras \textit{rokoroko} que a água levou, bem como as pacovãs, se
multiplicaram na terra dos \textit{napë}. Assim foi, as bananeiras se
multiplicaram. 

\chapter{Pore}
 
\letra{H}{apa}, ɨ̃naha të ã kua. Kamiyë pëma kɨ no patama rë pëtore hamɨ, kurata si
keaɨ taonomihe. Ɨhɨ të mao ha kë kunoha, kihi të si kɨ kuami. Ɨnaha kuwë
të si no pëtopɨrë mai! 

Ɨhɨ weti naha të ha taprarɨnɨ, kama a përɨopë ha, a përɨtopë ha, weti
naha a puhi ha kutarunɨ, kurata si kɨ kupropë të tama? Urihi pë kãi ma
rë humouwei, kihamɨ wa huɨ, poreĩxinoripɨ kama hawë ai të hikarimoma, të
ĩxino wararawë praa, praa hõkoa. Ɨnaha të rë kuawei ha hei kurenaha
kamanɨ ɨ̃hɨ hei kurenaha pëma hena pë tiyëpë. Kama Pore a rë përɨonowei,
ɨ̃hɨ heinaha tiyëwa e henakɨ kuoma. Hei kurenaha:

Ɨnaha henakɨ ta tiyëprarɨ! A noã tamoimi makui, ɨ̃hɨnɨ henakɨ kãi tiyëwa
kuoma, hei yãa kurenaha. A hiramonomi makui. Ɨhɨ a rë përɨre ha, a rë
përɨonowei ha, ĩxino kama e të ha tararɨnɨ, Pore kama ĩxinoripɨ he
harayoma. Ɨhɨ të pë kuopë ha, hawë kurata si pë rë kure, të pë tuku ma
rë xɨrɨkɨi, mokohe mo si pë rë kui. Ɨhɨ mo si kɨ a rë kuprarionowei,
kurata si kɨ. 

Porenɨ kama ĩxinoripɨ ha këarunɨ, kama poo enɨ, të pë ha pëarɨnɨ, ĩxino
ha pëarunɨ, të ha ĩximarɨnɨ, ɨ̃hɨ mokohe mosi kɨ ma kuonowei, kamanɨ a
keanomi. Mokohe mosi pë kuopë ha, tuku uprahaopë ha, të pëkema. Pëarɨnɨ,
të ĩximarema, ai të henaha, ai të henaha, kurata si kɨ.

Proto! Pauximɨ! Proto! Rokomɨ! Prohto! Monarimɨ! Proto! Pakatarimɨ!
Proto! Nakoaximɨ! Rokoya! Kama rokoroko e kɨ, roorewë, kama e xĩro
harayoma. Ɨhɨ mokohe mosi pë ĩximapë ha, ai të henaha, ai të henaha, të
mɨɨ mɨ ayoma. Hukomo ɨ̃ha e kãi homoprarioma. Ai të yahi ha, ai të yahi
ha, a ha yahirɨnɨ, a ha yurënɨ, a yuanomi. Ɨha kama Pore ni pëtopë ha,
kurata e si kɨ kupropë ha, mokohe e mosi kɨ kuratapropë ha, hukomo e
pëtarioma. Ɨhamɨ e kau homoprarioma. Ãhëãkɨ ɨ̃harë, kumawë ma kɨ ɨ̃harë,
prãki ãsi kɨ ɨ̃harë, ɨ̃naha të pë kuprarioma. Xamakoro e kãi kaurayoma.
Ɨharë ɨ̃hɨ Porenɨ kurata si kɨ rë pëtamarenowei kurata si kɨ. 

Pore ihamɨ si kɨ, ɨ̃hɨ exi të ha e si kɨ pëtarioma? Ihirupɨ e mahu kua
yaro. Suwë e kuami makui, wãro, ɨ̃naha e kuoma. Ihirupɨ e kua yaro,
kurata si kɨ pëtamarema, mokohe mosi kɨ kurataprarioma. 

Ɨhɨ Pore a rë kuinɨ si kɨ, ɨ̃hɨ iha si kɨ kararu piyërema, Horonamɨnɨ, a
he ha harënɨ. A përɨa ha tararɨnɨ, ihirupɨ a wã he ha harënɨ, ihirupɨ
mahu e të wai kuoma. Payeri kuonomi, suwë pë yaɨ ai yaɨ e kuonomi.
Yanomamɨ të pë xapopɨpropë, të pë xapopɨ hiraɨ ha. Mahu e të wai takema,
moko makui. Hei tëhë, a rë kuonowei naha, a hekura kuwëmi. 

Hëyëmɨ e ha kuaaimanɨ, hëyëha a he harema, a he hareyoruma. Ɨhɨ heinɨ a
he rë haarënɨ, kurata si kɨ kararu nakarema, Pore iha. Yanomamɨ të pë
hikaripɨ mao tëhë, të pë përɨo mao tëhë, ɨ̃hɨ iha si kɨ ha yurënɨ, të pë
ha hirakɨnɨ, kama e të pë ha hirakionɨ, a kõopë ha, të pë noã ha tarɨnɨ,
ɨ̃hɨ kurata si kɨ kãi wãha ha yuprarɨnɨ, si kararu kearemahe. Ɨhɨ pëma a
piyëmaɨ puhio yaro, si kɨ yurema. Kihamɨ si kɨ rë pëtono rë kure hamɨ,
ai ɨ̃ha si kɨ kua xoaa: rokomɨ, nakoaximɨ, rokoya, pauximɨ, si pë kua
xoaa. Ɨnaha të kuprarioma. 

Hei kamiyë pëma kɨnɨ no patama rë kui, pëma kɨ napë pë no patamapɨ rë
kuinɨ kurata a waɨ haɨonomihe. A wanomihe. Napë pë no patama maa xoaa
yaro. Kuami yaro. Ɨnaha të kuoma. Ɨhɨ weti iha kurata si kɨ rë yurehe,
si kɨ rë pararayonowei, weti a wãha hapa kua? Pore a yaia. Pore hesiopɨ
xo kɨ përɨpɨoma. Porakapɨ. Kutaenɨ ɨ̃hɨ iha a rë pararayonowei kurata,
napë pënɨ kurata a kãi taɨhe. 

Ɨhɨ weti naha si kɨ yua ha tarë hërɨnɨ, weti Yanomamɨ tënɨ si kɨ ha yurë
hërɨnɨ, napë pë ihamɨ si kɨ he haapehe, si kɨ kurayo hërɨma? Ai tënɨ si
kɨ yuanomi. Suwë a ha yɨpɨmorɨnɨ, a pesi prakema. Suwë a rë yɨpɨmore
hamɨ, mau unɨ suwë a ha puhinɨ, a riëhëwë yaro, ɨ̃hɨ exi të kuprarioma?
Suwë pë rë kui, exi të pë të kupropë? A yure hërɨma, a no ha puhiarɨnɨ.
Hei a suwë yɨpɨmono rë yurenowei, a no uhutipɨ pata u hamɨ a kurarioma.
Pruka a kuprarioma. Pei a yai. Kama unɨ. 

--- Suwë ya puhii! Xoape, tëëhë a ta hio! --- u pata ha kunɨ, kama u pata
harayoma. 

U pata haɨ nosi yauama. U pata haɨ xoatarioma. Kihi u pata, kiha të pata
ma tirere, kihi xĩka hamɨ të mɨ pata tëaaɨ he yatia. 

--- Prako! Prako! --- u pata kuma. 

Ɨhɨ tëhë hei pë tëë a rë kui a yãprarema, heinaxomi naha, të rë
kurenaha, hei kurenaha, a yãprama, heparapɨnɨ. Pë hɨɨnɨ e noã waxukema.
Kama a mixi no tukepɨ ha, mau unɨ a napë kuyëpraimaɨ yaro, yari a huɨ
tëhë, u pë pata rë kuaaɨwei naha, u pata kuaama. Hawë pë të u pata
hoyahoyamaɨhe, të u pata kuaaɨ ha, yãprano a kemaparema, a riëhëwë
makui. Ɨha pë hɨɨ e kepema. Pë tëë a rë kui ɨ̃hɨ unɨ e yure hërɨma.
Kõamaɨ kõanomi. A mixi kãi tuamanomi. Mau unɨ a yure herɨma, hesiopɨ. 

--- Hei mau ya u rë kui, ya u nɨɨpɨ kupropë, ya yurei kuhe --- e u kuma. 

Kuwë yaro pë nɨɨ e u kua, Yanomamɨ. U pata harayoma. A ha yãprarɨnɨ,
werehi e texinakɨ kãi huukema. A mɨ kãi yãakema, riëhëwë a oni
taprarema, wãima e henakɨ hoyaremahe, hoyaɨ tëhë e u pata hama, hei ipa
xapono kurenaha e kuoma. 

--- Nape! Nape! Nakami a ta yãprarɨxë! A ta pauxiprarɨxë! A ta pauxipraɨ
haɨro! 

--- Pëhë kɨ puhi kuaaɨ përai kë, xei, kuopëtao kë yaɨ wanɨ a ta hipëkɨxë!
--- mau u makui ha, pë hɨɨ e kuma. E hipëamaɨ puhima. Kama nomahẽa.
Ɨnaha e kuma. A amoa kua, mau unɨ a rë yure herɨnowei: 

--- Xiri tõi! Xiri tõi, xiri tõiwë, xiri tõi, xiri tõi, xiri tõi, xiri
tõiwë! --- e kurayoma. Ɨhɨ kama hẽaropɨ u wãha yuaɨ ha: 

--- Tuuuuuuuuu! --- a wã hurema, mau unɨ. Ɨhɨ pë hɨɨ: 

--- Xiri tõi! Xiri tõi! Xiri tõi! --- pë hɨɨ e kuma. 

Kuɨ tëhë, ɨ̃hɨ ei rë e të rë takɨhe ha, e të huhe taɨ tëhë, a pehi kãi
mori raɨa hërɨɨ tëhë, pë yaɨnɨ a xëyëparema, a hurihia nokarema, e mɨa
no preo makui. E ɨ̃kɨma, pë yaɨ a mɨa no poma. 

-Ɨɨaaaɨ̃ɨ! Apawë, apawë kuo pëtao! Hapemi, hapemi, kuo pëtao! Napemi,
napemi kuo pëtao! --- e kuma. 

A ma kuɨ tëhë, a hurihia he yatirema. 

--- Hɨ̃ɨ Kopou! --- a ëpëtarema. 

A xëyëa ëpëparema. Kuaaɨ tëhë, a nokare herɨma. Ɨhɨ a no tapomaɨ yaro, u
rë õkimohe, u õki rërëɨ makuimi. Hɨ̃ɨɨ! Urihi a pata! Puuuu! U pata kuma.
Heinaha të pata kutario hërɨma.

--- Aëëë! --- suwë a kutario hërɨma. 

Ɨhɨ a rë potuprarionowei, ɨ̃hɨ rë pë pokëkou, yɨpɨ a kemaparema yaro. A pesi praoma yaro, naka ĩyëo xoaoma, ĩyëĩyë hëyëmɨ e yõu xoawë yaro a
kemaparema. Kutaenɨ hei mau u nɨɨpɨ kuprarioma. Kama a no uhutipɨ, pë
huokema, pë xerereokema. Mau u kɨ haikirema. Ɨhɨ tëhë rokoroko si pë
pata rë yure herɨnowei, kurata ai pë pehi pata rë yure herɨnowei, pë
pararayoma, napë pë urihipɨ hamɨ! Ɨnaha të kuprarioma, paraomopotayoma.

 \chapter{A anta que andava nas árvores}
 
\letra{Foi}{Horonamɨ} quem perguntou os nomes dos animais. Horonamɨ encheu a
floresta de animais. 
Horonamɨ encontrou a anta Xamari, que andava como Yanomami. Ela andava
nos galhos baixos, vindo em sua direção. 

--- \textit{Hukru}! \textit{Hukru}! \textit{Prãããõ}! --- ela fez ao cair. 

Ela andava nas árvores como os cuatás. Afinal, ele encontrou a anta
andando nas árvores. Felizmente, ele fez com que ela descesse, para que
nós pudéssemos comê-la. 

É sempre um acontecimento quando matamos uma anta para comê-la! 

A anta não andava no chão: andava nas árvores de uma espécie nativa de
louro, atravessando os galhos e comendo as frutas maduras. Horonamɨ fez
quebrar o galho para que a anta caísse. Depois de cair, ela se acostumou
a andar no chão. 

A anta chegou ao xapono dos esquilos, mas lá não deu certo, então ela
foi para a mata. Os esquilos se juntaram quando a anta ainda era Yanomami,
e a chamaram. Queriam saber quanto ela aguentava comer.

Os esquilos viviam como Yanomami: moravam em um xapono no alto das
árvores e faziam festas como nós, embora eles fossem se tornar animais.
Um dia, eles chamaram as cutias, os caititus, as queixadas, as antas, os
papagaios e as maitacas. Havia muita comida, mas os convidados não
conseguiram comer tudo. Até a anta também desistiu de comer, pois
pressentiam que algo ia acontecer. 

De repente, todos eles se transformaram em animais. 

As queixadas também eram Yanomami. Os cipós se
arrebentaram e elas caíram. Foi lá, na região do xapono dos esquilos
onde não conseguiram comer, pois estavam prestes a se transformar. Não
havia nenhuma queixada antes de eles se transformarem. Nessas regiões,
não havia queixada. Subiram até o alto, subiram, estavam subindo até a
ponta do cipó. Lá, o cipó arrebentou no meio. Queixada! Se isso não
tivesse acontecido, lá naquela floresta, hoje as queixadas andariam nas
árvores. 

A anta foi quem caiu primeiro e passou a andar no chão, tornando-se um
animal terrestre. Em seguida, o cipó das queixadas arrebentou. Outros
Yanomami, que ficaram na parte superior do cipó se transformaram em
macacos cuatás. Assim foi. 

As queixadas ocuparam toda a floresta. Elas desceram rio abaixo.
Horonamɨ conseguiu assim fazer a anta descer ao chão, e hoje nós as
comemos. Assim que foi. Não havia animais no início, pois eles viviam
espalhados, como os Yanomami, em vários xaponos. 

\textit{Yãukuakua}! \textit{Yãukuakua}! Ninguém fazia assim. É assim mesmo. Esse
grande animal que anda no chão, quando estamos famintos de carne, nós a
comemos, ela anda mesmo no chão. Nós a comemos. 

\chapter{Xama a rë ɨmɨnowei}
 
\letra{Ɨ}{hɨnɨ} xĩro yaro a rë warirenɨ, ɨ̃hɨnɨ urihi a no yaropɨ kãi tapramarema. 

Xama a makui, a he kãi harema, Xamari Yanomamɨ a huma. Kihamɨ yahatoto
hamɨ a ɨmɨma, kiha të pë pata ɨmɨɨ: 

--- Hukru! Hukru! Prãaão! --- a pata ha prërënɨ, a pata kuma. 

Paxo kurenaha xama a ɨmɨma. A ɨmɨɨ he haa piyërema, hore kunomai, a kea
piyëmarema a horehewë tikowë yaro, xama, kamiyë pëma kɨnɨ pëma pë wapë. 

Yakumɨ pë ha niaprahenɨ pëma pë wapë. Kahu kɨ hamɨ a pata ha ɨmɨrɨ
hërɨnɨ, a pata ha piyëikunɨ, tatetate kɨ wapë. Ɨnaha xama pita hamɨ a
hunomi, hapa. Ɨmɨrewë kë a kuoma. Ɨhɨ a rë ɨmɨre, a pata ha kerɨnɨ, pita
hamɨ a hua xoarayoma. A hua hexipaa xoarayoma. 

Wayapaxiri pë iha a waroo xi ha wãrianɨ, urihi hamɨ a hurayoma. Ɨha a
kerayoma, a pehi ha këprarunɨ. Ɨhɨ kõmi të pë ha kõkaprarunɨ, Xamari a
Yanomamɨ kuo tëhë, a nakaremahe. A wausi wapapehe, Wayapaxiri pënɨ.

 Yanomamɨ pë hiraoma, xapono kurenaha pruka pë hiraoma, pë reahumoma.
Yaro pë kuoma makui, pë kãi reahumoma. Wayapaxi pë rë kui, tomɨ, poxe,
warë, xama, werehi, ãrima pë nakaa hɨtɨtɨrema. Makui, Wayapaxi pë ni
haikianomihe. Xama a makui, a kãi tɨraa no prekema. 

Ɨha pë xi rii wãrihou xoaoma. Warë Yanomamɨ pë kuoma. Ɨha pë pehi kãi
hëtɨmarema. Ɨharë Wayapaxiri pë iha pë iaɨ xi wãriama, warë a hunomi.
Hei pë urihi hamɨ warë pë hunomi. Ɨhɨ kihamɨ horehe hamɨ warë pë mori
ɨmɨma, hɨtɨtɨwë. ?hete hei pë ora pata rë tuore, të pë pata ɨmɨɨ, ora
pata kuaa hërɨɨ, hërɨɨ, hërɨɨ, kihi tokori pë rë kurati naha, kiha pë
pehi pata hëtɨrayoma. Warë! 

Xama xoma hamɨ a kerayoma, pita hamɨ a huɨ waikio tëhë, a pitamou waikio
tëhë, ɨ̃hɨ të nosi yau hamɨ warë pë pehi rë hëtɨre, paxo ai pë hurayoma.
Oraora paxo kë pë. Ɨnaha pë kuprarioma. 

Warë pë rë kui, hei pë pata rë hëtɨre, urihi a rë kui a haikiprarioma.
Hei pei pë koro yai rë kui pata u koro rë kure hamɨ pë pehi pata
nihõroye hërɨma. Hei pëma pë wapë. Ɨnaha të kuprarioma. Yaro a hunomi,
hapa, pë përɨhɨwë yaro, Yanomamɨ kurenaha të pë xaponopɨ kuprawë yaro,
pë hunomi. 

--- Yãukuakua! Yãukuakua! --- ai të pë kãi kunomi. Ɨnaha të yai kua. Ɨhɨ a
pata rë hure, a ha pitaprarunɨ, kamiyë pëma kɨ naikii, a wamopë a
pitapramaɨ he yatirayoma. Pëma a wapë. 